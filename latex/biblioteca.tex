\documentclass[11pt, a4paper, twoside]{article}
\usepackage[T1]{fontenc}
\usepackage[utf8]{inputenc}
\usepackage{amssymb,amsmath}
\usepackage[portuguese]{babel}
\usepackage{comment}
\usepackage{datetime}
\usepackage[pdfusetitle]{hyperref}
\usepackage[all]{xy}
\usepackage{graphicx}
\addtolength{\parskip}{.5\baselineskip}

%aqui comeca o que eu fiz de verdade, o resto veio e eu to com medo de tirar
\usepackage{xcolor}
\usepackage{listings} %biblioteca pro codigo
\usepackage{color}    %deixa o codigo colorido bonitinho
\usepackage[landscape, left=1cm, right=0.5cm, top=1cm, bottom=1.5cm]{geometry} %pra deixar a margem do jeito que o brasil gosta

\definecolor{gray}{rgb}{0.4, 0.4, 0.4} %cor pros comentarios
%\renewcommand{\footnotesize}{\small} %isso eh pra mudar o tamanho da fonte do codigo
\setlength{\columnseprule}{0.2pt} %barra separando as duas colunas
\setlength{\columnsep}{10pt} %distancia do texto ate a barra

\lstset{ %opcoes pro codigo
breaklines=true,
keywordstyle=\color{blue}\bfseries,
commentstyle=\color{gray},
breakatwhitespace=true,
language=C++,
%frame=single, % nao sei se gosto disso ou nao
numbers=none,
rulecolor=\color{black},
showstringspaces=false
stringstyle=\color{blue},
tabsize=4,
basicstyle=\ttfamily\footnotesize, % fonte
}
\lstset{literate=
%   *{0}{{{\color{red!20!violet}0}}}1
%    {1}{{{\color{red!20!violet}1}}}1
%    {2}{{{\color{red!20!violet}2}}}1
%    {3}{{{\color{red!20!violet}3}}}1
%    {4}{{{\color{red!20!violet}4}}}1
%    {5}{{{\color{red!20!violet}5}}}1
%    {6}{{{\color{red!20!violet}6}}}1
%    {7}{{{\color{red!20!violet}7}}}1
%    {8}{{{\color{red!20!violet}8}}}1
%    {9}{{{\color{red!20!violet}9}}}1
%	 {l}{$\text{l}$}1
	{~}{$\sim$}{1} % ~ bonitinho
}

\title{MWNWMWNNWMWNWN \\ USP}
\author{Nathan Luiz, Willian Mori e Willian Wang}


\begin{document}
\twocolumn
\date{\today}
\maketitle


\renewcommand{\contentsname}{Índice} %troca o nome do indice para indice
\tableofcontents


%%%%%%%%%%%%%%%%%%%%
%
% strings
%
%%%%%%%%%%%%%%%%%%%%

\section{strings}

\subsection{ Err Tree - Palindromic Tree}
\begin{lstlisting}
//  Description: A tree such that each node represents a 
//               palindrome of string s. It is possible to append
//               a character.
//  Complexity: Amortized O(|s|)
//

c71 struct palindromic_tree {
        // node info
320     vector<int> len, link, freq;
dd9     vector<map<char, int>> to;
    
d92     int cur, cnt;
35a     palindromic_tree (int N): cur(1), cnt(2), len (N), link (N), to (N), freq (N) { len[0] = -1; }
ea5     void add(int i, string& s) {
f11         int p = len[cur] == i ? link[cur] : cur;
763         while (s[i - len[p] - 1] != s[i]) p = link[p];
060         if (to[p].find(s[i]) != to[p].end()) cur = to[p][s[i]];
4e6         else {
e6f             int L = link[p];
7e6             while (s[i - len[L] - 1] != s[i]) L = link[L];
9b8             L = max<int>(1, to[L][s[i]]);
46f             cur = to[p][s[i]] = cnt;
65c             len[cnt] = len[p] + 2; link[cnt] = L;
f65             cnt++;
60b         }
54f         freq[cur]++;
1e1     }
bfc     void insert(string& s) {
724         for (int i = 0; i < s.size (); i++)
df9             add(i, s);
251     }
ed4 };
\end{lstlisting}

\subsection{ String Hashing}
\begin{lstlisting}

//  Functions:
//      str_hash - Builds the hash in O(|S|)
//      operator() - Gives the number representing substring s[l,r] in O(1)

//  Details:
//          - To use more than one prime, you may use long long, __int128 or array<int>
//          - You may easily change it to handle vector<int> instead of string
//          - Other large primes: 1000041323, 100663319, 201326611
//          - If smaller primes are needed(For instance, need to store the mods in an array):
//              - 50331653, 12582917, 6291469, 3145739, 1572869
//

4ba const long long mod1 = 1000015553, mod2 = 1000028537;

878 mt19937 rng((int) chrono::steady_clock::now().time_since_epoch().count()); // random number generator

463 int uniform(int l, int r) {
a7f     uniform_int_distribution<int> uid(l, r);
f54     return uid(rng);
d9e }

3fb template<int MOD> 
d7d struct str_hash {
c63     static int P;
dcf     vector<ll> h, p;
0e1     str_hash () {}
ea8     str_hash(string s) : h(s.size()), p(s.size()) {
7a2         p[0] = 1, h[0] = s[0];
ad7         for (int i = 1; i < s.size(); i++)
84c             p[i] = p[i - 1]*P%MOD, h[i] = (h[i - 1]*P + s[i])%MOD;
1ef     }
af7     ll operator()(int l, int r) { // retorna hash s[l...r]
749         ll hash = h[r] - (l ? h[l - 1]*p[r - l + 1]%MOD : 0);
dfd         return hash < 0 ? hash + MOD : hash;
3ba     }
977 };
217 template<int MOD> int str_hash<MOD>::P = uniform(256, MOD - 1); // l > |sigma|

61c struct Hash {
        // Uses 2 primes to better avoid colisions
3b6     str_hash<mod1> H1;
b36     str_hash<mod2> H2;
     
e3d     Hash (string s) : H1(str_hash<mod1>(s)), H2(str_hash<mod2>(s)) {}
     
af7     ll operator()(int l, int r) {
f6f         ll ret1 = H1(l, r), ret2 = H2(l, r);
742         return (ret1 << 30) ^ (ret2);
d2e     }
b31 };
\end{lstlisting}

\subsection{KMP}
\begin{lstlisting}
// mathcing(s, t) retorna os indices das ocorrencias
// de s em t
// autKMP constroi o automato do KMP

// Complexidades:
// pi - O(n)
// match - O(n + m)
// construir o automato - O(|sigma|*n)
// n = |padrao| e m = |texto|

0a1 template <typename T> vector<int> kmp(int sz, const T s[]) {
924 	vector<int> pi(sz);
e8d 	for(int i=1;i<sz;i++) {
730 		int &j = pi[i];
6ef 		for(j=pi[i-1];j>0 && s[i]!=s[j];j=pi[j-1]);
04b 		if(s[i] == s[j]) j++;
4fb 	}
81d 	return pi;
b29 };

c10 template<typename T> vector<int> matching(T& s, T& t) {
658 	vector<int> p = pi(s), match;
a1b 	for (int i = 0, j = 0; i < t.size(); i++) {
6be 		while (j and s[j] != t[i]) j = p[j-1];
c4d 		if (s[j] == t[i]) j++;
310 		if (j == s.size()) match.push_back(i-j+1), j = p[j-1];
028 	}
ed8 	return match;
c82 }

a2d struct KMPaut : vector<vector<int>> {
47c 	KMPaut(){}
6c7 	KMPaut (string& s) : vector<vector<int>>(26, vector<int>(s.size()+1)) {
503 		vector<int> p = pi(s);
04b 		auto& aut = *this;
4fa 		aut[s[0]-'a'][0] = 1;
19a 		for (char c = 0; c < 26; c++)
5d3 			for (int i = 1; i <= s.size(); i++)
42b 				aut[c][i] = s[i]-'a' == c ? i+1 : aut[c][p[i-1]];
4bb 	}
79b };
\end{lstlisting}

\subsection{Suffix Array}
\begin{lstlisting}
// Description: Algorithm that sorts the suffixes of a string.
//              The last character of the string to be sorted 
//              must be the smallest!
// Complexity: O(|s| log(|s|)).
//

// Suffix Array da KTH
3f4 struct SuffixArray {
ac0     string s;
716     vector<int> sa, lcp;
264     SuffixArray () {}
91c     SuffixArray(string v, int lim=256) { // or basic_string<int>
3ef         s = v;
861         int n = s.size(), k = 0, a, b;
99c         vector<int> x(all(s)+1), y(n), ws(max(n, lim)), rank(n);
8a6         sa = lcp = y; iota(all(sa), 0);
25d         for (int j = 0, p = 0; p < n; j = max(1, j * 2), lim = p) {
e59             p = j; iota(all(y), n - j);
3fc             for(int i = 0; i < n; i++) if (sa[i] >= j) y[p++] = sa[i] - j;
911             fill(all(ws), 0);
483             for(int i = 0; i < n; i++) ws[x[i]]++;
5d9             for(int i = 1; i < lim; i++) ws[i] += ws[i - 1];
a9e             for (int i = n; i--;) sa[--ws[x[y[i]]]] = y[i];
3ff             swap(x, y); p = 1; x[sa[0]] = 0;
b7e             for(int i = 1; i < n; i++) a = sa[i - 1], b = sa[i], x[b] =
da0                 (y[a] == y[b] && y[a + j] == y[b + j]) ? p - 1 : p++;
4b4         }
9c7         for (int i = 1; i < n; i++) rank[sa[i]] = i;
05c         for (int i = 0, j; i < n - 1; lcp[rank[i++]] = k)
9f6             for (k && k--, j = sa[rank[i] - 1]; s[i + k] == s[j + k]; k++);
289     }
3e5 };
\end{lstlisting}

\subsection{Z}
\begin{lstlisting}
1a8 	vector<int> ret(sz);
6ed 	for(int l=0,r=0,i=1;i<sz;i++) {
52d 		auto expand = [&]() {
568 			while(r<sz && s[r-l]==s[r]) r++;
38b 			ret[i] = r-l;
8a3 		};
08f 		if(i >= r) {
018 			l=r=i;
eec 			expand();
9d9 		} else {
bb7 			if(ret[i-l] < r-i) ret[i] = ret[i-l];
4e6 			else {
537 				l=i;
eec 				expand();
c99 			}
48c 		}
5d0 	}
edf 	return ret;
ad3 };
\end{lstlisting}



%%%%%%%%%%%%%%%%%%%%
%
% data-structures
%
%%%%%%%%%%%%%%%%%%%%

\section{data-structures}

\subsection{ MO algorithm}
\begin{lstlisting}
//  Description: 
//      Answers queries offline with sqrt decomposition.
//  Complexity: 
//      exec - O(n*sqrt(n)*O(remove / add))
d90 const int magic = 230;

670 struct Query {
738     int l, r, idx;
9a6     Query () {}
e7d     Query (int _l, int _r, int _idx) : l(_l), r(_r), idx(_idx) {}
9ae     bool operator < (const Query &o) const {
2a8         return mp(l / magic, r) < mp(o.l / magic, o.r);
717     }
d25 };

5ce struct MO {
8d9     int sum;
55c     MO(vector<ll> &v) : sum(0), v(v), cnt(N), C(N) {}
    
fe9     void exec(vector<Query> &queries, vector<ll> &answers) {
14d         answers.resize(queries.size());
bfa         sort(queries.begin(), queries.end());
    
3df         int cur_l = 0;
cf5         int cur_r = -1;
    
275         for (Query q : queries) {
71e             while (cur_l > q.l) {
ec6                 cur_l--;
939                 add(cur_l);
60c             }
294             while (cur_r < q.r) {
bda                 cur_r++;
d95                 add(cur_r);
c3b             }
b32             while (cur_l < q.l) {
631                 remove(cur_l);
cf9                 cur_l++;
ddf             }
6eb             while (cur_r > q.r) {
198                 remove(cur_r);
99e                 cur_r--;
d76             }
553             answers[q.idx] = get_answer(cur_l, cur_r);
8bc         }
dce     }
    
c96     void add(int i) {
683         sum += v[i];
0c3     }
    
17e     void remove(int i) {
f2f         sum -= v[i];
9a0     }
    
3b1     ll get_answer(int l, int r) {
e66         return sum;
520     }
3f7 };
\end{lstlisting}

\subsection{ Search Buckets}
\begin{lstlisting}
//  Data structure that provides two operations on an array:
//  1) set array[i] = x
//  2) count how many i in [start, end) satisfy array[i] < value
//  Both operations take sqrt(N log N) time. Amazingly, because of 
//  the cache efficiency this is faster than the(log N)^2 algorithm 
//  until N = 2-5 million.

39d template<typename T> struct search_buckets {
        // values are just the values in order. buckets are sorted in segments of BUCKET_SIZE (last segment may be smaller)
931     int N, BUCKET_SIZE;
c8c     vector<T> values, buckets;
    
9f1     search_buckets(const vector<T> &initial = {}) {
b48         init(initial);
611     }
    
7d4     int get_bucket_end(int bucket_start) const {
5e4         return min(bucket_start + BUCKET_SIZE, N);
0e2     }
    
ac2     void init(const vector<T> &initial) {
51b         values = buckets = initial;
2e7         N = values.size();
ecf         BUCKET_SIZE = 3 * sqrt(N * log(N + 1)) + 1;
8bb         cerr << "Bucket size: " << BUCKET_SIZE << endl;
    
2fc         for (int start = 0; start < N; start += BUCKET_SIZE)
23b             sort(buckets.begin() + start, buckets.begin() + get_bucket_end(start));
167     }
    
89e     int bucket_less_than(int bucket_start, T value) const {
8b6         auto begin = buckets.begin() + bucket_start;
188         auto end = buckets.begin() + get_bucket_end(bucket_start);
6b9         return lower_bound(begin, end, value) - begin;
21f     }
    
92e     int less_than(int start, int end, T value) const {
b52         int count = 0;
23a         int bucket_start = start - start % BUCKET_SIZE;
23c         int bucket_end = min(get_bucket_end(bucket_start), end);
    
93c         if (start - bucket_start < bucket_end - start) {
af4             while (start > bucket_start)
d53                 count -= values[--start] < value;
9d9         } else {
ad3             while (start < bucket_end)
62d                 count += values[start++] < value;
358         }
    
590         if (start == end)
308             return count;
    
655         bucket_start = end - end % BUCKET_SIZE;
e51         bucket_end = get_bucket_end(bucket_start);
    
23c         if (end - bucket_start < bucket_end - end) {
ec0             while (end > bucket_start)
807                 count += values[--end] < value;
9d9         } else {
612             while (end < bucket_end)
8da                 count -= values[end++] < value;
250         }
    
7bf         while (start < end && get_bucket_end(start) <= end) {
395             count += bucket_less_than(start, value);
a28             start = get_bucket_end(start);
5b1         }
    
c08         assert(start == end);
308         return count;
4cf     }
    
ea5     int prefix_less_than(int n, T value) const {
629         return less_than(0, n, value);
e45     }
    
2c3     void modify(int index, T value) {
985         int bucket_start = index - index % BUCKET_SIZE;
e50         int old_pos = bucket_start + bucket_less_than(bucket_start, values[index]);
48b         int new_pos = bucket_start + bucket_less_than(bucket_start, value);
    
85e         if (old_pos < new_pos) {
30f             copy(buckets.begin() + old_pos + 1, buckets.begin() + new_pos, buckets.begin() + old_pos);
8b8             new_pos--;
                // memmove(&buckets[old_pos], &buckets[old_pos + 1], (new_pos - old_pos) * sizeof(T));
9d9         } else {
670             copy_backward(buckets.begin() + new_pos, buckets.begin() + old_pos, buckets.begin() + old_pos + 1);
                // memmove(&buckets[new_pos + 1], &buckets[new_pos], (old_pos - new_pos) * sizeof(T));
b97         }
    
cac         buckets[new_pos] = value;
9cf         values[index] = value;
54b     }
ec7 };
\end{lstlisting}

\subsection{ Segtree 2D}
\begin{lstlisting}
//
//  Complexity:
//      build - O(N)
//      query - O(logN^2)
//  

// struct Node {
//     Node () {}
//     Node operator + (const Node &o) const{
//         return Node ();
//     }
// };

4c1 namespace Seg2D {
14e     int n,m;
2ea     Node a[MAXN][MAXN], st[2*MAXN][2*MAXN];
    
b45     Node op (Node a, Node b){
534         return a + b;
978     }
    
0a8     void build (){
6e0         for(int i = 0; i < n; i++) for(int j = 0; j < m; j++)st[i+n][j+m]=a[i][j];
034         for(int i = 0; i < n; i++) for(int j = m - 1; j; --j)
c2d             st[i + n][j] = op(st[i + n][j << 1],st[i + n][j << 1 | 1]);
61e         for(int i = n - 1; i; --i) for(int j = 0; j < 2 * m; j++)
da1             st[i][j]=op(st[2 * i][j], st[2 * i + 1][j]);
de2     }
82b     void upd (int x, int y, Node v){
365         st[x + n][y + m] = v;
2e7         for(int j = y + m; j > 1; j /= 2) st[x + n][j / 2] = op(st[x + n][j], st[x + n][j ^ 1]);
eac         for(int i = x + n; i > 1; i /= 2) for(int j = y + m; j; j /= 2)
aa4             st[i / 2][j] = op(st[i][j], st[i ^ 1][j]);
12a     }
    
        // essa query vai de x0, y0 ate x1 - 1, y1 - 1 !!!
243     Node query (int x0, int x1, int y0, int y1) {
2ae         Node r = Node (); // definir elemento neutro da query!!!
6a8         for (int i0 = x0 + n, i1 = x1 + n; i0 < i1; i0 /= 2, i1 /= 2){
0b4             int t[4], q = 0;
f0e             if (i0 & 1) t[q++] = i0++;
847             if (i1 & 1) t[q++] = --i1;
18d             for(int k = 0; k < q; k++) for(int j0 = y0 + m, j1 = y1 + m; j0 < j1; j0 /= 2,j1 /= 2){
acb                 if (j0 & 1) r = op(r, st[t[k]][j0++]);
401                 if (j1 & 1) r = op(r, st[t[k]][--j1]);
a9d             }
3cd         }
4c1         return r;
33a     }
388 };
\end{lstlisting}

\subsection{ Sparse Segtree 2D}
\begin{lstlisting}
//  Grid of dimensions N x M
//
//  Operations:
//          update(x, y, val) <- update on point (x, y)
//          query(lx, rx, ly, ry) <- query on rectangle [lx..rx] x [ly..ry]
//
//  O(logNlogM) complexity per operation
//  O(N + UlogNlogM) memory, where U is the number of updates
//
//  Possible changes:
//      - Speed: Use iterative segment tree or BIT on N axis
//      - O(UlogNlogM) memory: Make N axis sparse too
//

b5b namespace seg2d {
    	// YOU ONLY NEED TO CHANGE THIS BLOCK
9a8 	const int N = 200'000, M = 200'000;
0cb 	using T = int32_t;
0ce 	const T zero = 0; // INF if maintaining minimum, for example
cad 	T merge(T a, T b) {
534 		return a + b;
7f7 	}
    
bf2 	struct Node {
9fa 		T s = zero;
8d9 		int32_t l = 0, r = 0;
09f 	};
28a 	int root[4*N];
afe 	vector<Node> v;
    
288 	void upd(int& no, int l, int r, int pos, T val) {
270 		if(not no) {
2ec 			no = v.size();
    			//assert(no < v.capacity());
903 			v.emplace_back();
74e 		}
ad4 		if(l == r) v[no].s = val; // !!! OR v[no].s = merge(v[no].s, val) !!!
4e6 		else {
ee4 			int m = (l+r)/2;
611 			auto &[s, nl, nr] = v[no];
303 			if(pos <= m) upd(nl, l, m, pos, val);
926 			else upd(nr, m+1, r, pos, val);
064 			s = merge(v[nl].s, v[nr].s);
c01 		}
741 	}
    
a21 	T qry(int no, int l, int r, int ql, int qr) {
3c6 		if(not no) return zero;
966 		if(qr < l || r < ql) return zero;
611 		auto &[s, nl, nr] = v[no];
856 		if(ql <= l && r <= qr) return s;
ee4 		int m = (l+r)/2;
84f 		return merge(qry(nl, l, m, ql, qr),
a48 				qry(nr, m+1, r, ql, qr));
eb6 	}
    
389 	void upd(int no, int l, int r, int x, int y, T val) {
30a 		upd(root[no], 0, M-1, y, val);
8ce 		if(l == r) return;
ee4 		int m = (l+r)/2;
410 		if(x <= m) upd(2*no, l, m, x, y, val);
1c3 		else upd(2*no+1, m+1, r, x, y, val);
a50 	}
    
89a 	T qry(int no, int l, int r, int lx, int rx, int ly, int ry) {
8db 		if(rx < l || r < lx) return zero;
060 		if(lx <= l && r <= rx) return qry(root[no], 0, M-1, ly, ry);
ee4 		int m = (l+r)/2;
019 		return merge( qry(2*no, l, m, lx, rx, ly, ry),
d11 				qry(2*no+1, m+1, r, lx, rx, ly, ry) );
4df 	}
    
153 	void build(int no, int l, int r) {
fee 		root[no] = v.size();
903 		v.emplace_back();
8ce 		if(l == r) return;
ee4 		int m = (l + r) / 2;
b4b 		build(2*no, l, m);
4d7 		build(2*no+1, m+1, r);
88e 	}
    
8d3 	void update(int x, int y, T val) {
561 		upd(1, 0, N-1, x, y, val);
96e 	}
    
fad 	int query(int lx, int rx, int ly, int ry) {
4c7 		return qry(1, 0, N-1, lx, rx, ly, ry);
82c 	}
    
    	// receives max number of updates
    	// each update creates at most logN logM nodes
    	// RTE if we reserve less than number of nodes created
977 	void init(int maxu) {
618 		v.reserve(400*maxu);
903 		v.emplace_back();
826 		build(1, 0, N-1);
466 	}
00e }
\end{lstlisting}

\subsection{Binary Indexed Tree}
\begin{lstlisting}
// !! zero indexed !!
// all operations are O(logN)

273 template<typename T> struct Bit {
678 	vector<T> bit;
052 	Bit(int n): bit(n) {}
    
f3c 	void update(int id, T val) {
bd2 		for(id+=1; id<=int(bit.size()); id+=id&-id)
28c 			bit[id-1] += val;
5bb 	}
    
32d 	T query(int id) {
e86 		T sum = T();
2d6 		for(id+=1; id>0; id-=id&-id)
fee 			sum += bit[id-1];
e66 		return sum;
dd6 	}
    
    	// returns the first prefix for which sum of 0..=pos >= val
    	// returns bit.size() if such prefix doesnt exists
    	// it is necessary that v[i] >= 0 for all i for monotonicity
ccc 	int lower_bound(T val) {
e86 		T sum = T();
bec 		int pos = 0;
7f2 		int logn = 31 - __builtin_clz(bit.size());
a99 		for(int i=logn;i>=0;i--) {
148 			if(pos + (1<<i) <= int(bit.size()) 
8f3 					&& sum + bit[pos + (1<<i) - 1] < val) {
b1b 				sum += bit[pos + (1<<i) - 1];
b2c 				pos += (1<<i);
7ba 			}
8f9 		}
d75 		return pos;
e0c 	}
e4f };
\end{lstlisting}

\subsection{Binary Indexed Tree 2D}
\begin{lstlisting}
// 0-indexed
// update(x, y, val): m[row][col] += val
// query(x, y): returns sum m[0..=x][0..=y]
ecd template <typename T> struct Bit2D {
14e 	int n, m;
678 	vector<T> bit;
26f 	Bit2D(int _n, int _m): n(_n), m(_m), bit(n*m) {}
    
848 	T query(int x, int y) {
19a 		T res = 0;
ab3 		for(x+=1;x>0;x-=x&-x)
aad 			for(int z=y+1;z>0;z-=z&-z)
50c 				res += bit[(x-1)*m+z-1];
b50 		return res;
a3e 	}
    
8d3 	void update(int x, int y, T val) {
157 		for(x+=1;x<=n;x+=x&-x)
a36 			for(int z=y+1;z<=m;z+=z&-z)
522 				bit[(x-1)*m+z-1] += val;
08d 	}
5c8 };
\end{lstlisting}

\subsection{Implicit Lazy Treap}
\begin{lstlisting}
// All operations are O(log N)
// If changes need to be made in lazy propagation,
// See Node::push()
// To extend behavior of Treap, use inheritance
//
// Important functions:
// Treap::insert(int ind, T info)
// Treap::erase(int ind)
// Treap::reverse(int l, int r)
// Treap::operator[](int ind)

798 mt19937_64 rng(chrono::steady_clock::now().time_since_epoch().count());

a2c template <typename ND, typename T = typename ND::T> 
125 struct Treap {
691 	ND *root;
006 	vector<ND> v;
    	// max: maximum number of NDs created
1be 	Treap(int max): root(0) {
890 		v.reserve(max);
d0e 	}
21d 	ND* new_ND(T info) {
    		// assert(v.size() != v.capacity());
6b6 		v.emplace_back(info);
ec2 		return &v.back();
18e 	}
f6e 	int getl(ND& nd) {
892 		return nd.l ? nd.l->sz : 0;
9c5 	}
3bb 	void merge(ND* l, ND* r, ND*& res) {
9b5 		if(!l || !r) {
0e8 			res = l ? l : r;
505 			return;
1e7 		}
48f 		l->push(); r->push();
e4a 		if(l->h > r->h) {
8ee 			res = l;
d17 			merge(l->r, r, l->r);
9d9 		} else {
516 			res = r;
c88 			merge(l, r->l, r->l);
40e 		}
7d2 		res->pull();
c4e 	}
    	// left treap has size pos
c8d 	void split(ND* x, ND*& l, ND*& r, int pos, int ra = 0) {
91e 		if(!x) {
c77 			l = r = 0;
505 			return;
d3d 		}
77d 		x->push();
bac 		int nra = ra + getl(*x) + 1;
f4f 		if(pos < nra) {
f45 			split(x->l, l, r, pos, ra);
14a 			x->l = r;
2ec 			r = x;
9d9 		} else {
6cb 			split(x->r, l, r, pos, nra);
4b0 			x->r = l;
954 			l = x;
544 		}
c32 		x->pull();
732 	}
    	// Merges all s and makes them root
3da 	template <int SZ>
6ac 	void merge(array<ND*, SZ> s) {
815 		root = 0;
bf7 		for(ND* nd: s)
abe 			merge(root, nd, root);
467 	}
    	// Splits root into SZ EXCLUSIVE intervals
    	// [0..s[0]), [s[0]..s[1]), [s[1]..s[2])... [s[SZ-1]..end)
    	// Example: split<3>({l, r}) gets the exclusive interval [l, r)
3da 	template <int SZ> 
c7e 	array<ND*, SZ> split(array<int, SZ-1> s) {
f43 		assert(s.back() <= (root ? root->sz : 0));
d09 		array<ND*, SZ> res;
dc9 		split(root, res[0], res[1], s[0]);
de9 		for(int i=1;i<SZ-1;i++)
291 			split(res[i], res[i], res[i+1], s[i]-s[i-1]);
815 		root = 0;
b50 		return res;
907 	}
4b4 	void insert(int ind, T info) {
488 		auto s = split<2>({ind});
974 		merge<3>({s[0], new_ND(info), s[1]});
517 	}
97a 	void erase(int ind) {
4e4 		auto s = split<3>({ind, ind+1});
e6f 		merge<2>({s[0], s[2]});
00b 	}
420 	T operator[](int ind) {
68b 		assert(0 <= ind && ind < root->sz);
e88 		ND* x = root;
77d 		x->push();
7ce 		for(int ra = 0, nra = getl(*x); nra != ind; nra = ra + getl(*x)) {
7f3 			if(nra < ind) ra = nra + 1, x = x->r;
303 			else x = x->l;
77d 			x->push();
d19 		}
79c 		return x->info;
b5e 	}
87e };

bf2 struct Node {
06a 	using T = int;
247 	T info;
c17 	Node *l, *r;
e4d 	int sz;
5fa 	uint64_t h;
    	// more fields here
e19 	Node(T i): info(i), l(0), r(0), sz(1), h(rng()), plus(0) {}
247 	void push() {}
48b 	void pull() {
a26 		sz = 1;
9f5 		for(auto c: {l, r}) 
90a 			if(c) sz += c->sz;
335 	}
8f0 };

362 struct MyTreap : Treap<Node> {
0aa 	MyTreap(int max): Treap<Node>(max) {}
    	// new methods here
8b8 };
\end{lstlisting}

\subsection{Iterative Segment Tree}
\begin{lstlisting}
// Supports non-commutative operations
//
// functions:
// 	update(pos, val): set leaf node in pos to val
// 	query(l, r): get sum of nodes in l and r
//
// Example: Range minimum queries segtree:
// 	struct Node {
// 		using T = int;
// 		T mn;
// 		Node(): mn(numeric_limits<T>::max()) {}
// 		Node(T x): mn(x) {}
// 		friend Node operator+(Node lhs, Node rhs) {
// 			return Node(min(lhs.mn, rhs.mn));
// 		}
// 	};
// 	using SegMin = SegIt<Node>;
//
//	int main() {
//		vector<int> v{3,1,3};
// 		SegMin seg(v);
// 		assert(seg.query(0, 2).mn == 1);
// 		seg.update(1, 5);
// 		assert(seg.query(0, 2).mn == 3);
// 		assert(seg.query(1, 1).mn == 5);
// 	}
//
// Submission: https://codeforces.com/contest/380/submission/193484078

a2c template <typename ND, typename T = typename ND::T>
2a0 struct SegIt {
1a8 	int n;
c50 	vector<ND> t;
    
0d6 	SegIt(int _n): n(_n), t(2*n) {} 
681 	SegIt(vector<T> &v): n(v.size()), t(2*n) {
830 		for(int i=0;i<n;i++)
766 			t[i+n] = ND(v[i]);
6f2 		build();
20d 	}
    
0a8 	void build() {
917 		for(int i=n-1;i>0;i--) 
f23 			t[i] = t[2*i] + t[2*i+1];
6b1 	}
    
6a3 	void update(int pos, T val) {
f11 		int p = pos + n;
5e3 		t[p] = ND(val);
d08 		while(p) {
d31 			p /= 2;
6c7 			t[p] = t[2*p] + t[2*p+1];
05a 		}
283 	}
    
a64 	ND query(int l, int r) {
844 		ND tl, tr;
e5f 		r++; // to make query inclusive
4f7 		for(l += n, r += n; l < r; l /= 2, r /= 2) {
e91 			if(l&1) tl = tl + t[l++];
ae4 			if(r&1) tr = t[--r] + tr;
c73 		}
cf9 		return tl + tr;
efd 	}
7d4 };
\end{lstlisting}

\subsection{NCE}
\begin{lstlisting}
// op(l, i), op(r, i) = true if they exist
// l = -1, r = v.size() otherwise
//
// Example: nce(v, greater<T>()): for each i returns
// nce[i] = {
//  biggest l < i such that v[l] > v[i]
//  smallest r > i such that v[r] > v[i]
// }
//
// Complexity: O(N)

751 template <typename T, typename OP>
101 vector<pair<int, int>> nce(vector<T> v, OP op) {
3d2 	int n = v.size();
a3d 	vector<pair<int, int>> res(n);
fd9 	vector<pair<T, int>> st;
603 	for(int i=0;i<n;i++) {
195 		while(!st.empty() && !op(st.back().first, v[i])) 
d73 			st.pop_back();
a33 		if(st.empty()) res[i].first = -1;
53d 		else res[i].first = st.back().second;
e89 		st.emplace_back(v[i], i);
cdc 	}
23e 	st.clear();
45b 	for(int i=n-1;i>=0;i--) {
195 		while(!st.empty() && !op(st.back().first, v[i])) 
d73 			st.pop_back();
0b7 		if(st.empty()) res[i].second = n;
ce3 		else res[i].second = st.back().second;
e89 		st.emplace_back(v[i], i);
ba8 	}
b50 	return res;
793 }
\end{lstlisting}

\subsection{Ordered Set}
\begin{lstlisting}
30f #include <ext/pb_ds/tree_policy.hpp>
0d7 using namespace __gnu_pbds;
// iterator find_by_order(size_t index), size_t order_of_key(T key)
67a template <typename T> 
994 using ordered_set=tree<T, null_type, less<T>, rb_tree_tag, tree_order_statistics_node_update>;
\end{lstlisting}

\subsection{Persistent segment tree.}
\begin{lstlisting}
// Complexity: 
//      Update - O(logn) memory and time
//      Query - O(logn) time
// Lazy works, but is very expensive. Every query will need O(logn) memory.

c35 template<class T, int SZ> struct pseg {
ec3     static const int LIMIT = 1e7; // adjust
6a0     int nex;
b0a     vector<int> l, r;
5ea     vector<T> val, lazy;
9da     pseg () : l (LIMIT), r (LIMIT), val (LIMIT), nex (0) {}
    
a69     int copy(int cur) {
269         int x = nex++;
5d0         val[x] = val[cur], l[x] = l[cur], r[x] = r[cur]; // lazy[x] = lazy[cur];
ea5         return x;
c0e     }
f57     T comb(T a, T b) { return a + b; }
c85     void pull(int x) { val[x] = comb (val[l[x]], val[r[x]]); } 
    //  void push(int cur, int L, int R) { 
    //      if (!lazy[cur]) return;
    //      if (L != R) {
    //          l[cur] = copy(l[cur]);
    //          val[l[cur]] += lazy[cur];
    //          lazy[l[cur]] += lazy[cur];
    //          
    //          r[cur] = copy(r[cur]);
    //          val[r[cur]] += lazy[cur];
    //          lazy[r[cur]] += lazy[cur];
    //      }
    //      lazy[cur] = 0;
    //  }
     
        //// MAIN FUNCTIONS
e73     T query(int cur, int lo, int hi, int L, int R) {  
e3f         if (lo <= L && R <= hi) return val[cur];
65a         if (R < lo || hi < L) return 0;
331         int M = (L+R)/2;
fb1         return comb(query(l[cur],lo,hi,L,M), query(r[cur],lo,hi,M+1,R));
e5b     }
14c     int upd(int cur, int pos, T v, int L, int R) {
b63         if (R < pos || pos < L) return cur;
    
dc1         int x = copy(cur);
7e8         if (pos <= L && R <= pos) { val[x] = v; return x; }
            
331         int M = (L+R)/2;
6e0         l[x] = upd(l[x],pos,v,L,M), r[x] = upd(r[x],pos,v,M+1,R);
d65         pull(x); return x;
89f     }
4eb     int build(vector<T>& arr, int L, int R) {
6a9         int cur = nex++;
651         if (L == R) {
d8c             if (L < (int) arr.size ()) val[cur] = arr[L];
6c5             else val[cur] = T ();
75e             return cur;
85d         }
    
331         int M = (L+R)/2;
3a7         l[cur] = build(arr,L,M), r[cur] = build(arr,M+1,R);
c36         pull(cur); return cur;
1ba     }
        
        //// PUBLIC
b3e     vector<int> loc;
4e6     void upd(int pos, T v) { loc.pb (upd (loc.back(), pos, v, 0, SZ-1)); }
048     T query(int ti, int lo, int hi) { return query (loc[ti], lo, hi ,0 , SZ - 1); }
fa1     void build(vector<T>& arr) { loc.pb (build(arr, 0, SZ - 1)); }
554 };
\end{lstlisting}

\subsection{RMQ}
\begin{lstlisting}
//      Answers queries on a range.
//  Complexity: 
// 		build - O(N logN)
// 		query - O(1)

f81 template <typename T> struct RMQ {
572 	vector<vector<T>> dp;
6bc 	T ops(T a, T b) { return min(a,b); }
fae 	RMQ() {}
f16 	RMQ(vector<T> v) {
3d2 		int n = v.size();
1e7 		int log = 32-__builtin_clz(n);
ca2 		dp.assign(log, vector<T>(n));
79c 		copy(all(v), dp[0].begin());
738 		for(int l=1;l<log;l++) for(int i=0;i<n;i++) {
447 			auto &cur = dp[l], &ant = dp[l-1];
c4e 			cur[i] = ops(ant[i], ant[min(i+(1<<(l-1)), n-1)]);
c57 		}
ec3 	}
0ad 	T query(int a, int b)  {
90f 		if(a == b) return dp[0][a];
6a7 		int p = 31-__builtin_clz(b-a);
dd7 		auto &cur = dp[p];
ec5 		return ops(cur[a], cur[b-(1<<p)+1]);
089 	}
386 };
\end{lstlisting}



%%%%%%%%%%%%%%%%%%%%
%
% flow-and-matching
%
%%%%%%%%%%%%%%%%%%%%

\section{flow-and-matching}

\subsection{Blossom}
\begin{lstlisting}
// If white[v] = 0 after solve() returns, v is part of every max matching
//
// Complexity: O(NM), faster in practice

9f5 struct MaxMatching {
1a8 	int n;
903 	vector<vector<int>> adj;
dfd 	vector<int> mate, first;
026 	vector<bool> white;
a6e 	vector<pair<int,int>> label;
    
ca0 	MaxMatching(int _n): n(_n), adj(n+1), mate(n+1), first(n+1), white(n+1), label(n+1) {}
    
c22 	void add_edge(int u, int v) { adj[u].pb(v); adj[v].pb(u); }
    
e8b 	int group(int x) { int &f = first[x]; return white[f] ? f = group(f) : f; }
    
761 	void match(int p, int b) {
3db 		swap(b, mate[p]); if(mate[b] != p) return;
134 		auto [f, s] = label[p];
c3c 		if(!s) mate[b] = f, match(f, b); // vertex label
bef 		else match(f, s), match(s, f); // edge label
a1f 	}
    
b99 	bool augment(int st) { // assert(st);
3c6 		white[st] = 1; first[st] = 0; label[st] = {0,0};
132 		queue<int> q; q.push(st);
14d 		while (!q.empty()) {
20f 			int a = q.front(); q.pop(); // outer vertex
bae 			for(auto &b: adj[a]) { // assert(b);
870 				if (white[b]) { // two outer vertices, form blossom
a46 					int x = group(a), y = group(b), lca = 0;
d2c 					while(x || y) {
2e1 						if(y) swap(x, y);
791 						if(label[x] == pair<int,int>{a, b}) { lca = x; break; }
3ff 						label[x] = {a,b}; x = group(label[mate[x]].first);
068 					}
77e 					for (int v: {group(a), group(b)}) while (v != lca) {
    						// assert(!white[v]); // make everything along path white
908 						q.push(v); white[v] = true; first[v] = lca;
89b 						v = group(label[mate[v]].first);
795 					}
f91 				} else if (!mate[b]) { // found augmenting path
304 					mate[b] = a; match(a, b); fill(all(white), false); // reset
8a6 					return true;
b8d 				} else if (!white[mate[b]]) {
cac 					white[mate[b]] = true; first[mate[b]] = b;
66b 					label[b] = {0,0}; label[mate[b]] = {a, 0};
277 					q.push(mate[b]);
63f 				}
247 			}
a3e 		}
d1f 		return false;
4ae 	}
f5c 	int solve() {
1a4 		int ans = 0;
32a 		for(int st=1; st<=n; st++) if(!mate[st]) ans += augment(st);
    		// for(int st=1; st<=n; st++) if(!mate[st] && !white[st]) assert(!augment(st));
ba7 		return ans;
10a 	}
dfb };
\end{lstlisting}

\subsection{Dinitz}
\begin{lstlisting}
// get_flow(s, t): Returns max flow with source s and sink t
//
// Complexity: O(E*V^2). If unit edges only: O(E*sqrt(V))

14d struct Dinic {
670 	struct edge {
b7a 		int to, cap, flow;
0e3 	};
    
789 	vector<vector<int>> g;
1e7 	vector<int> lvl;
37c 	vector<edge> e;
    
db3 	Dinic(int sz): g(sz), lvl(sz) {}
    
233 	void add_edge(int s, int t, int cap) {
1f3 		int id = e.size();
ffd 		g[s].push_back(id);
614 		e.push_back({t, cap, 0});
634 		g[t].push_back(++id);
ff7 		e.push_back({s, cap, cap});
8e0 	}
    
123 	bool bfs(int s, int t) {
5c1 		fill(all(lvl), INF);
0d6 		lvl[s] = 0;
26a 		queue<int> q;
08b 		q.push(s);
f76 		while(!q.empty() && lvl[t] == INF) {
b1e 			int v = q.front();
833 			q.pop();
ca6 			for(int id: g[v]) {
5c7 				auto [p, cap, flow] = e[id];
bd9 				if(lvl[p] != INF || cap == flow)
5e2 					continue;
ed5 				lvl[p] = lvl[v] + 1;
00a 				q.push(p);
e2f 			}
e19 		}
8de 		return lvl[t] != INF;
c9d 	}
    
2b1 	int dfs(int v, int pool, int t, vector<int>& st) {
23a 		if(!pool) return 0;
413 		if(v == t) return pool;
138 		for(;st[v]<(int)g[v].size();st[v]++) {
59b 			int id = g[v][st[v]];
56f 			auto &[p, cap, flow] = e[id];
783 			if(lvl[v]+1 != lvl[p] || cap == flow) continue;
1de 			int f = dfs(p, min(cap-flow, pool) , t, st);
235 			if(f) {
c87 				flow += f;
ef4 				e[id^1].flow -= f;
abe 				return f;
964 			}
e0b 		}
bb3 		return 0;
7a0 	}
    
704 	int get_flow(int s, int t) {
    		//reset to initial state
    		//for(int i=0;i<e.size();i++) e[i].flow = (i&1) ? e[i].cap : 0;
11e 		int res = 0;
678 		vector<int> start(g.size());
8ce 		while(bfs(s,t)) {
cb6 			fill(all(start), 0);
449 			while(int f = dfs(s,INF,t,start)) 
5f5 				res += f;
7a9 		}
b50 		return res;
c83 	}
a7c };
\end{lstlisting}

\subsection{Hungarian}
\begin{lstlisting}
// Resolve o problema de assignment (matriz n x n)
// Colocar os valores da matriz em 'a' (pode < 0)
// assignment() retorna um par com o valor do
// assignment minimo, e a coluna escolhida por cada linha
//
// O(n^3)

513 template<typename T> struct Hungarian {
c04 	static constexpr T INF = numeric_limits<T>::max();
1a8 	int n;
a08 	vector<vector<T>> a;
f36 	vector<T> u, v;
5ff 	vector<int> p, way;
    
0e9 	Hungarian(int n_) : n(n_), a(n, vector<T>(n)), u(n+1), v(n+1), p(n+1), way(n+1) {}
    
40e 	void set(int i, int j, T w) { a[i][j] = w; }
    
d67 	pair<T, vector<int>> assignment() {
78a 		for (int i = 1; i <= n; i++) {
8c9 			p[0] = i;
625 			int j0 = 0;
f49 			vector<T> minv(n+1, INF);
0c1 			vector<bool> used(n+1);
016 			do {
472 				used[j0] = true;
d24 				int i0 = p[j0], j1 = -1;
8bc 				T delta = INF;
9ac 				for (int j = 1; j <= n; j++) if (!used[j]) {
7bf 					T cur = a[i0-1][j-1] - u[i0] - v[j];
9f2 					if (cur < minv[j]) minv[j] = cur, way[j] = j0;
821 					if (minv[j] < delta) delta = minv[j], j1 = j;
4d1 				}
f63 				for (int j = 0; j <= n; j++)
2c5 					if (used[j]) u[p[j]] += delta, v[j] -= delta;
6ec 					else minv[j] -= delta;
6d4 				j0 = j1;
52a 			} while (p[j0] != 0);
016 			do {
4c5 				int j1 = way[j0];
0d7 				p[j0] = p[j1];
6d4 				j0 = j1;
886 			} while (j0);
431 		}
306 		vector<int> ans(n);
6db 		for (int j = 1; j <= n; j++) ans[p[j]-1] = j-1;
def 		return {-v[0], ans};
06e 	}
7b6 };
\end{lstlisting}

\subsection{Mincost Max-Flow}
\begin{lstlisting}
// shortest paths. Useful when the edges costs are negative. 
// Infinite loop if there's a negative cycle.
//
// Constructor:
// MinCost(n, s, t)
// n - number of nodes in the flow graph.
// s - source of the flow graph.
// t - sink of the flow graph.
//
// Methods:
// - add_edge(u, v, cap, cost)
//   adds a directed edge from u to v with capacity `cap` and cost `cost`.
// - get_flow()
//   returns a pair of integers in which the first value is the maximum flow and the
//   second is the minimum cost to achieve this flow.
//
// Complexity: There are two upper bounds to the time complexity of getFlow
//              - O(max_flow * (E log V))
//              - O(V * E * (E log V))

cfd struct MinCost {
cd7 	static constexpr int INF = 1e18;
670 	struct edge {
22a 		int to, next, cap, cost;
30a 	};
748 	int n, s, t;
439 	vector<int> first, prev, dist;
70d 	vector<bool> queued;
93b 	vector<edge> g;
    
10d 	MinCost(int _n, int _s, int _t) : n(_n), s(_s), t(_t),
52d 		first(n, -1), prev(n), dist(n), queued(n) {};
    
5cb 	void add_edge(int u, int v, int cap, int cost) {
270 		int id = g.size();
4a6 		g.pb({v, first[u], cap, cost});
c19 		first[u] = id;
3a5 		g.pb({u, first[v], 0, -cost});
727 		first[v] = ++id;
b65 	}
    
cbc 	bool augment() {
04e 		fill(all(dist), INF);
a93 		dist[s] = 0;
0d9 		queued[s] = 1;
26a 		queue<int> q;
08b 		q.push(s);
14d 		while(!q.empty()) {
e4a 			int u = q.front(); 
833 			q.pop();
a04 			queued[u] = 0;
ba2 			for(int e = first[u]; e != -1; e = g[e].next) {
17a 				int v = g[e].to;
762 				int ndist = dist[u] + g[e].cost;
de7 				if(g[e].cap > 0 && ndist < dist[v]) {
d72 					dist[v] = ndist;
20e 					prev[v] = e;
076 					if(!queued[v]) {
2a1 						q.push(v);
b84 						queued[v] = 1;
67c 					}
90d 				}
cd6 			}
a9a 		}
85d 		return dist[t] < INF;
    		//UNCOMMENT FOR MIN COST WITH ANY FLOW (NOT NECESSARILY MAXIMUM)
    		//return dist[t] <= 0;
cc6 	}
    
a53 	pair<int, int> get_flow() {
05f 		int flow = 0, cost = 0;
456 		while(augment()) {
612 			int cur = t, curf = INF;
9c2 			while(cur != s) {
a51 				int e = prev[cur];
887 				curf = min(curf, g[e].cap);
58c 				cur = g[e^1].to;
574 			}
8bc 			flow += curf; 
1fc 			cost += dist[t] * curf;
cd6 			cur = t;
9c2 			while(cur != s) {
a51 				int e = prev[cur];
09b 				g[e].cap -= curf;
787 				g[e^1].cap += curf;
58c 				cur = g[e^1].to;
765 			}
24b 		}
884 		return {flow, cost};
42b 	}
9e2 };
\end{lstlisting}



%%%%%%%%%%%%%%%%%%%%
%
% problems
%
%%%%%%%%%%%%%%%%%%%%

\section{problems}

\subsection{ LIS-2D}
\begin{lstlisting}
//      Given N pairs of numbers, find the lenght of the biggest
//      sequence such that a_i < a_i+1, b_i<b_i+1
//  Complexity: 
//      O(N (logN)^2)
//  Details:
//      It uses divide & conquer with a segtree to make all
//      comparisons fast. memo[i] contains the answer for 
//      the biggest sequence ending in i.
//  0eb0fc
//

093 const int N = 2e5 + 10;

2ad int n, memo[N];
89b pair<int, int> a[N];

a2f struct segTree {
1a8     int n;
2e6     vector<ll> st;
aac     ll combine(ll a, ll b) {
a16         return max (a, b); // TODO define merge operator
d19     }
401     segTree() {}
8d5     segTree(int n) : n (n), st (2 * n, -1) {}
625     void update(int i, ll x) {
cbf         st[i += n] = max (x, st[i + n]); // TODO change update operation
b8f         while (i > 1) {
29a             i >>= 1;
4f9             st[i] = combine(st[i << 1], st[i << 1 | 1]);
479         }
16b     }
        // query from l to r, inclusive
02a     ll query(int l, int r) {
721         ll resl = -1, resr = -1;
326         for (l += n, r += n+1; l < r; l >>= 1, r >>= 1) {
ced             if (l & 1) resl = combine(resl, st[l++]);
386             if (r & 1) resr = combine(st[--r], resr);
97c         }
f9d         return combine(resl, resr);
4a1     }
220 };


6cb void divide_conquer (int l, int r) {
8ce     if (l == r) return;
ee4     int m = (l + r) / 2;
917     divide_conquer (l, m); // calculamos o valor para esquerda
    
        // propagamos para a direita
        // temos que comprimir coordenadas
f2a     vector<int> M;
b08     for (int j = l; j <= m; j++) {
ee6         M.push_back (a[j].first + 1);
153         M.push_back (a[j].second);
d22     }
38c     for (int j = m + 1; j <= r; j++) {
cd2         M.push_back (a[j].first);
153         M.push_back (a[j].second);
1d0     }
862     sort (all (M));
8fd     unique (all (M));
078     auto find_pos = [&] (int x) {
23d         return (int) (lower_bound (all (M), x) - M.begin ());
bae     };
    
ea3     vector<array<int, 4>> events;
        // coord_x, L/R, coord_y, memo/ind
917     for (int j = l; j <= m; j++)
64b         events.pb ({find_pos(a[j].first + 1), 0, find_pos(a[j].second), memo[j]});
ed6     for (int j = m + 1; j <= r; j++)
992         events.pb ({find_pos(a[j].first), 1, find_pos(a[j].second), j});
bb0     sort (all (events));
    
e5d     segTree st (M.size () + 1);
653     for (auto [x, op, y, M] : events) {
5f7         if (op == 0) st.update (y, M);
4e2         else memo[M] = max (memo[M], st.query (0, y - 1) + 1);
e82     }
7cf     divide_conquer (m + 1, r); // calculamos o valor para direita
76b }
\end{lstlisting}



%%%%%%%%%%%%%%%%%%%%
%
% math
%
%%%%%%%%%%%%%%%%%%%%

\section{math}

\subsection{ Coprimes}
\begin{lstlisting}
//      Given a set o integers, calculates the quantity of integers
//      in the set coprimes with x. You can actually make queries on
//      anything related to the coprimes. For example, sum of comprimes.
//  Complexity: 
//      precalc - O(n logn)
//      add - O(sigma(N))
//      coprime - O(sigma(N))
//  Details:
//      It uses Mobius Function. To add or remove an integer of the set
//      just change sign to +1 or -1.

49c struct Coprimes {
1a8     int n;
bae     vector<ll> cnt;
afe     vector<int> U;
74f     vector<vector<int>> fat;
bbe     Coprimes () {}
7bb     Coprimes (int n) : n(n), U(n), fat(n), cnt(n) {
e91         precalc ();
67b     }
fe8     void precalc () {
9cf         U[1] = 1;
f65         for (int i = 1; i < n; i++) fat[i].pb (1);
6f5         for (int i = 1; i < n; i++) {
2ef             for (int j = 2 * i; j < n; j += i) U[j] -= U[i];
850             if (fat[i].size () == 1 && i > 1) {
1ec                 for (int j = i; j < n; j += i)
c25                     for (int k = fat[j].size () - 1; k >= 0; k--) 
d6d                         fat[j].pb (i * fat[j][k]);
62c             }
100         }
ab4     }
2f1     void add(int x, int sign){
2ed         for(auto d : fat[x]) cnt[d] += sign;
37f     }
a33     ll coprimo(int x){
92b         ll quant = 0;
41d         for(auto d : fat[x]){
903             quant += U[d] * cnt[d];
3b0         }
2bd         return quant;
0ce     }
1dc };
\end{lstlisting}

\subsection{ Gauss elimination - modulo 2}
\begin{lstlisting}
//  Description: 
//      Solves a linear system with n equations and m - 1 variables.
//      Is faster duo to the use of bitset.
//  Complexity: O(n^2 * m / 32)


//  Details:
//      Function solve return a boolean indicating if system is possible
//      or not. Also, if it is possible, the parameter maintains the answer.

146 struct Gauss_mod2 {
14e     int n, m;
66e     array<int, M> pos;
3e5     int rank = 0;
75f     vector<bitset<M>> a;
     
        // n equations, m-1 variables, last column is for coefficients
616     Gauss_mod2(int n, int m, vector<bitset<M>> &a): n(n), m(m), a(a) {
e55         pos.fill(-1);
eac     }
     
728     int solve(bitset<M> &ans) {
a73         for (int col = 0, row = 0; col < m && row < n; col++) {
896             int one = -1;
016             for (int i = row; i < n; i++) {
e6e                 if (a[i][col]) {
7ba                     one = i;
c2b                     break;
dff                 }
edb             }
     
b1a             if (one == -1) { continue; }
     
5fb             swap(a[one], a[row]);
     
8a0             pos[col] = row;
     
79f             for (int i = row + 1; i < n; i++) {
505                 if (a[i][col])
95d                     a[i] ^= a[row];
ecc             }
616             ++row, ++rank;
400         }
     
d16         ans.reset();
     
ca1         for (int i = m - 1; i >= 0; i--) {
413             if (pos[i] == -1) ans[i] = true;
4e6             else {
ec8                 int k = pos[i];
322                 for (int j = i + 1; j < m; j++) if (a[k][j]) ans[i] = ans[i] ^ ans[j];
506                 ans[i] = ans[i] ^ a[k][m];
4cc             }
e3a         }
     
332         for (int i = rank; i < n; i++) if (a[i][m]) return 0;
     
6a5         return 1;
dfa     }
f27 };
\end{lstlisting}

\subsection{ Gauss Xor - Gauss elimination mod 2}
\begin{lstlisting}
//               maintains a basis of the set.
//  Complexity: query - O(D)
//              add - O(D)

//  Functions:
//      query(mask) - returns the biggest number that can
//                    be made if you initially have cur and
//                    it cannot be bigger than lim.
//      add(mask) - adds mask to the basis.

//  Details:
//      We are assuming the vectors have size D <= 64. For general
//      case, you may change ll basis[] for bitset<D> basis[].

189 const int logN = 30;

3d7 struct Gauss_xor {
387     int basis[logN];
c9f     Gauss_xor () { memset (basis, 0, sizeof (basis)); }
     
5b8     void add (int x) {
a28         for (int j = logN - 1; j >= 0; j--) {
be0             if (x & (1ll << j)) {
335                 if (basis[j]) x ^= basis[j];
4e6                 else {
467                     basis[j] = x;
505                     return;
681                 }
58a             }
3f8         }
78e     }
     
cbd     int query (int j, int cur, int lim, bool mn) {
bfc         if (j < 0) return cur;
c5f         if (mn) {
5ad             return query (j - 1, max (cur, cur ^ basis[j]), lim, 1);
ec1         }
4e6         else {
9bc             if (lim & (1ll << j)) {
2c4                 if (cur & (1ll << j)) {
8ee                     int res = query (j - 1, cur, lim, 0);
d1e                     if (res) return res;
a86                 }
4e6                 else {
05a                     if (basis[j]) {
4ea                         int res = query (j - 1, cur ^ basis[j], lim, 0);
d1e                         if (res) return res;
591                     }
7d9                 }
ce3                 int val = min (cur, cur ^ basis[j]);
e2d                 if ((val & (1ll << j)) == 0) return query (j - 1, val, lim, 1);
98b                 else return 0;
39a             }
4e6             else {
2c4                 if (cur & (1ll << j)) {
e5f                     if (!basis[j]) return 0;
7c0                 }
12a                 return query (j - 1, min (cur, cur ^ basis[j]), lim, 0);
a33             }
651         }
bda     }
5db };
\end{lstlisting}

\subsection{ NTT - Number Theoretic Transform}
\begin{lstlisting}
//  Complexity: O(N logN)

//  Functions:
//      multiply(a, b)

//  Details:
//      Not all primes can be used and p = 998244353 is the most used prime. 
//      To multiply it for a general modulus, use 3 different possible primes 
//      and use Chinese Remainder Theorem to get the answear.

//  Possibilities
//  { 7340033, 5, 4404020, 1 << 20 },
//  { 415236097, 73362476, 247718523, 1 << 22 },
//  { 463470593, 428228038, 182429, 1 << 21},
//  { 998244353, 15311432, 469870224, 1 << 23 },
//  { 918552577, 86995699, 324602258, 1 << 22 }

ea8 namespace NTT {
7e5     using Z = mint<998244353>;
a92     const Z root(15311432), root_1(469870224);
32f     int root_pw = 1<<23;
    
506     void fft(vector<Z> & a, bool invert) {
94d 	int n = a.size();
    
5f9         for (int i = 1, j = 0; i < n; i++) {
4af             int bit = n >> 1;
474             for (; j & bit; bit >>= 1)
53c                 j ^= bit;
53c             j ^= bit;
aa5             if (i < j) swap(a[i], a[j]);
f3a         }
    
eb7         for (int len = 2; len <= n; len <<= 1) {
cf9             Z wlen = invert ? root_1 : root;
5ae             for (int i = len; i < root_pw; i <<= 1)
fe1                 wlen *= wlen;
    
6c8             for (int i = 0; i < n; i += len) {
973                 Z w(1);
2ae                 for (int j = 0; j < len / 2; j++) {
80c                     Z u = a[i+j], v = a[i+j+len/2] * w;
6c3                     a[i+j] = u + v;
273                     a[i+j+len/2] = u - v;
3e4                     w *= wlen;
6f5                 }
092             }
0da         }
    
eb5         if (invert) {
c61             Z n_1 = Z(n).inv();
bdf             for (Z & x : a) x *= n_1;
ff8         }
9e7     }
     
2e8     vector<Z> multiply(vector<Z> &a, vector<Z> &b) {
2a8         vector<Z> fa = a, fb = b;
015         int sz = a.size() + b.size() - 1, n = 1;
4ba         while (n < sz) n <<= 1;
    
75e         fa.resize(n), fb.resize(n);
404         fft(fa, 0), fft(fb, 0);
991         for (int i = 0; i < fa.size(); i++) fa[i] *= fb[i];
    
e55         fft(fa, 1);
c5c         fa.resize(sz);
83d         return fa;
61f     }
bf1 };
\end{lstlisting}

\subsection{Bit iterator}
\begin{lstlisting}
// use: for(auto it: BitIterator(n,m) { int mask = *it; ... }

368 struct BitIterator {
41c 	struct Mask {
f79 		uint32_t msk;
5f7 		Mask(uint32_t _msk): msk(_msk) {}
22e 		bool operator!=(const Mask& rhs) const { return msk < rhs.msk; };
29f 		void operator++(){const uint32_t t=msk|(msk-1);msk=(t+1)|(((~t&-~t)-1)>>__builtin_ffs(msk));}
600 		uint32_t operator*() const { return msk; }
cc7 	};
1dc 	uint32_t n, m;
75a 	BitIterator(uint32_t _n, uint32_t _m): n(_n), m(_m) {}
17a 	Mask begin() const { return Mask((1<<m)-1); }
d0d 	Mask end() const { return Mask((1<<n)); }
8ca };
\end{lstlisting}

\subsection{Convolutions}
\begin{lstlisting}
// Complexity: O(N logN)

// Functions:
//     multiply(a, b)
//     multiply_mod(a, b, m) - return answer modulo m

// Details:
//     For function multiply_mod, any modulo can be used. 
//     It is implemented using the technique of dividing 
//     in sqrt to use less fft. Function multiply may have
//     precision problems.
//     This code is faster than normal. So you may use it
//     if TL e tight.

d32 const double PI=acos(-1.0);
35b namespace fft {
3b2     struct num {
662         double x,y;
c0a         num() {x = y = 0;}
6da         num(double x,double y): x(x), y(y){}
cd4     };
4d4     inline num operator+(num a, num b) {return num(a.x + b.x, a.y + b.y);}
f7b     inline num operator-(num a, num b) {return num(a.x - b.x, a.y - b.y);}
b7b     inline num operator*(num a, num b) {
9f0         return num(a.x * b.x - a.y * b.y, a.x * b.y + a.y * b.x);
d63     }
db0     inline num conj(num a) {return num(a.x, -a.y);}
     
b58     int base = 1;
e47     vector<num> roots={{0,0}, {1,0}};
8a4     vector<ll> rev={0, 1};
148     const double PI=acosl(-1.0);
     
        // always try to increase the base
d50     void ensure_base(int nbase) {
11e         if(nbase <= base) return;
49f         rev.resize(1 << nbase);
55a         for (int i = 0; i < (1 << nbase); i++)
19e             rev[i] = (rev[i>>1] >> 1) + ((i&1) << (nbase-1));
2b8         roots.resize(1<<nbase);
775         while(base<nbase) {
21f             double angle = 2*PI / (1<<(base+1));
8cf             for(int i = 1<<(base-1); i < (1<<base); i++) {
52a                 roots[i<<1] = roots[i];
aef                 double angle_i = angle * (2*i+1-(1<<base));
922                 roots[(i<<1)+1] = num(cos(angle_i),sin(angle_i));
958             }
96d             base++;
af4         }
ae9     }
     
b94     void fft(vector<num> &a,int n=-1) {
05e         if(n==-1) n=a.size();
421         assert((n&(n-1)) == 0);
2fd         int zeros = __builtin_ctz(n);
a02         ensure_base(zeros);
4fa         int shift = base - zeros;
603         for (int i = 0; i < n; i++) {
3fc             if(i < (rev[i] >> shift)) {
b8b                 swap(a[i],a[rev[i] >> shift]);
9ac             }
b97         }
7cd         for(int k = 1; k < n; k <<= 1) {
cda             for(int i = 0; i < n; i += 2*k) {
0c2                 for(int j = 0; j < k; j++) {
d85                     num z = a[i+j+k] * roots[j+k];
20a                     a[i+j+k] = a[i+j] - z;
c9a                     a[i+j] = a[i+j] + z;
ee1                 }
804             }
b62         }
382     }
     
ba5     vector<num> fa, fb;
        // multiply with less fft by using complex numbers.
318     vector<ll> multiply(vector<ll> &a, vector<ll> &b);
     
        // using the technique of dividing in sqrt to use less fft.
966     vector<ll> multiply_mod(vector<ll> &a, vector<ll> &b, ll m, ll eq=0);
754     vector<ll> square_mod(vector<ll>&a, ll m);
7a3 };

// 16be45
7b3 vector<ll> fft::multiply(vector<ll> &a, vector<ll> &b) {
fe9     int need = a.size() + b.size() - 1;
    
217     int nbase = 0;
8da     while((1 << nbase) < need) nbase++;
4e5     ensure_base(nbase);
    
729     int sz = 1 << nbase;
9db     if(sz > (int)fa.size()) fa.resize(sz);
887     for(int i = 0; i < sz; i++) {
422         ll x = (i < (int)a.size() ? a[i] : 0);
435         ll y = (i < (int)b.size() ? b[i] : 0);
685         fa[i] = num(x, y);
3e3     }
    
650     fft(fa, sz);
4db     num r(0,-0.25/sz);
3ec     for(int i = 0; i <= (sz>>1); i++) {
b13         int j = (sz-i) & (sz-1);
afc         num z = (fa[j] * fa[j] - conj(fa[i] * fa[i])) * r;
f07         if(i != j) fa[j] = (fa[i] * fa[i] - conj(fa[j] * fa[j])) * r;
386         fa[i] = z;
488     }
    
650     fft(fa, sz);
5e0     vector<ll> res(need);
07f     for(int i = 0; i < need; i++) res[i] = fa[i].x + 0.5;
b50     return res;
16b }

// 4eb347
d99 vector<ll> fft::multiply_mod(vector<ll> &a, vector<ll> &b, ll m, ll eq) {
fe9     int need = a.size() + b.size() - 1;
217     int nbase = 0;
8da     while((1 << nbase) < need) nbase++;
4e5     ensure_base(nbase);
729     int sz = 1 << nbase;
9db     if(sz > (int)fa.size()) fa.resize(sz);
c0e     for(int i = 0; i < (int)a.size(); i++) {
538         ll x = (a[i] % m + m) % m;
7e5         fa[i] = num(x & ((1 << 15) - 1), x >> 15);
b60     }
26e     fill(fa.begin() + a.size(), fa.begin() + sz, num{0,0});
650     fft(fa, sz);
32a     if(sz > (int)fb.size()) fb.resize(sz);
b19     if(eq) copy(fa.begin(), fa.begin() + sz, fb.begin());
4e6     else {
1da         for(int i = 0; i < (int)b.size(); i++) {
044             ll x = (b[i] % m + m) % m;
418             fb[i] = num(x & ((1 << 15) - 1), x >> 15);
9f0         }
535         fill(fb.begin() + b.size(), fb.begin() + sz, num{0,0});
07e         fft(fb,sz);
59c     }
df3     double ratio = 0.25 / sz;
dc2     num r2(0, -1), r3(ratio, 0), r4(0, -ratio), r5(0,1);
3ec     for(int i = 0; i <= (sz>>1); i++) {
b13         int j = (sz - i) & (sz - 1);
d96         num a1 = (fa[i] + conj(fa[j]));
6c3         num a2 = (fa[i] - conj(fa[j])) * r2;
712         num b1 = (fb[i] + conj(fb[j])) * r3;
e45         num b2 = (fb[i] - conj(fb[j])) * r4;
41e         if(i != j) {
123             num c1 = (fa[j] + conj(fa[i]));
7ce             num c2 = (fa[j] - conj(fa[i])) * r2;
92b             num d1 = (fb[j] + conj(fb[i])) * r3;
a76             num d2 = (fb[j] - conj(fb[i])) * r4;
35f             fa[i] = c1 * d1 + c2 * d2 * r5;
525             fb[i] = c1 * d2 + c2 * d1;
55a         }
dc5         fa[j] = a1 * b1 + a2 * b2 * r5;
d92         fb[j] = a1 * b2 + a2 * b1;
dc3     }
f68     fft(fa, sz); fft(fb, sz);
5e0     vector<ll> res(need);
ae6     for(int i = 0; i < need; i++) {
6e0         ll aa = fa[i].x + 0.5;
bb7         ll bb = fb[i].x + 0.5;
0ee         ll cc = fa[i].y + 0.5;
407         res[i] = (aa + ((bb%m) << 15) + ((cc%m) << 30))%m;
0d6     }
b50     return res;
ca4 }

b86 vector<ll> fft::square_mod(vector<ll> &a, ll m) {
dfb     return multiply_mod(a, a, m, 1);
dde }
\end{lstlisting}

\subsection{Dirichlet Trick}
\begin{lstlisting}
//      Find the partial sum of a multiplicative function.
//      This code works for Phi or Mobius functions.
// Details:
//      It is necessary to precalculate the values of at least
//      sqrt (N). But, the optimal value might be around N^(2/3)

04a namespace Dirichlet {
d9a     vector<int> f;
ce9     map<int,int> mp;
    
4ce     void init (vector<int> &mul_func) {
8dc         f.resize (mul_func.size ());
6d7         for (int i = 1; i < mul_func.size (); i++) f[i] = f[i - 1] + mul_func[i];
5ef     }
    
cfc     int calc (int x) {
486         if(x<=N) return f[x];
c4e         if(mp.find(x)!=mp.end()) return mp[x];
651         int ans = x * (x + 1) / 2;
166         for(int i = 2, r; i <= x; i = r + 1) {
37f             r=x/(x/i);
2a9             ans -= calc(x/i)*(r-i+1);
624         }
b6f         return mp[x]=ans;
3b5     }
187 }
\end{lstlisting}

\subsection{Extended gcd}
\begin{lstlisting}
5d9 pair<int,int> egcd(int a, int b) {
02b 	if(b == 0) return {1, 0};
60e 	auto [x, y] = egcd(b, a%b);
f07 	return {y, x - y * (a/b)};
5e5 }
\end{lstlisting}

\subsection{Fast Walsh-Hadamard trasform}
\begin{lstlisting}
// 	- op(a, b) = a "xor" b, a "or" b, a "and" b
// Complexity: O(n log n)

29a const ll N = 1<<20;

67a template <typename T>
372 struct FWHT {
a69 	void fwht(T io[], ll n) {
495 		for (ll d = 1; d < n; d <<= 1) {
b98 			for (ll i = 0, m = d<<1; i < n; i += m) {
499 				for (ll j = 0; j < d; j++) { /// Don't forget modulo if required
bdc 					T x = io[i+j], y = io[i+j+d];
703 					io[i+j] = (x+y), io[i+j+d] = (x-y);	// xor
    					// io[i+j] = x+y; // and
    					// io[i+j+d] = x+y; // or
afa 				}
7f3 			}
cf6 		}
fe6 	}
1f8 	void ufwht(T io[], ll n) {
495 		for (ll d = 1; d < n; d <<= 1) {
b98 			for (ll i = 0, m = d<<1; i < n; i += m) {
499 				for (ll j = 0; j < d; j++) { /// Don't forget modulo if required
bdc 					T x = io[i+j], y = io[i+j+d];
    					 /// Modular inverse if required here
174 					io[i+j] = (x+y)>>1, io[i+j+d] = (x-y)>>1; // xor
    					// io[i+j] = x-y; // and
    					// io[i+j+d] = y-x; // or
027 				}
711 			}
fd4 		}
e69 	}
    	// a, b are two polynomials and n is size which is power of two
683 	void convolution(T a[], T b[], ll n) {
26b 		fwht(a, n), fwht(b, n);
e95 		for (ll i = 0; i < n; i++)
33e 			a[i] = a[i]*b[i];
23c 		ufwht(a, n);
5b4 	}
    	// for a*a	
f9a 	void self_convolution(T a[], ll n) {
d85 		fwht(a, n);
e95 		for (ll i = 0; i < n; i++)
35c 			a[i] = a[i]*a[i];
23c 		ufwht(a, n);
4c8 	}
ba3 };
d0d FWHT<ll> fwht;
\end{lstlisting}

\subsection{Gauss elimination}
\begin{lstlisting}

76a using arr = valarray<double>;
320 struct Gauss {
14e 	int n, m;
146 	vector<arr> v;
b2f 	Gauss(int _n, int _m): n(_n), m(_m), v(n, arr(m)) {}
    
d5f 	arr& operator[](int i) { return v[i]; }
    
958 	void eliminate() {
    		// eliminate column j
8ec 		for(int j=0;j<min(n, m);j++) {
2e9 			v[j].swap(*max_element(v.begin()+j, v.end(), 
73a 				[&](arr& a, arr& b) { return abs(a[j]) < abs(b[j]); }));
8c0 			for(int i=j+1;i<n;i++)
f68 				v[i] -= v[i][j] / v[j][j] * v[j];
06c 		}
b9c 	}
acf };
\end{lstlisting}

\subsection{Mint}
\begin{lstlisting}
e54 struct mint {
3ec 	int x;
9f5 	mint(): x(0) {}
609 	mint(int _x): x(_x%MOD<0?_x%MOD+MOD:_x%MOD) {}
5b8 	void operator+=(mint rhs) { x+=rhs.x; if(x>=MOD) x-=MOD; }
a9a 	void operator-=(mint rhs) { x-=rhs.x; if(x<0)x+=MOD; }
d08 	void operator*=(mint rhs) { x*=rhs.x; x%=MOD; }
152 	void operator/=(mint rhs) { *this *= rhs.inv(); }
9a2 	mint operator+(mint rhs) { mint res=*this; res+=rhs; return res; }
ee4 	mint operator-(mint rhs) { mint res=*this; res-=rhs; return res; }
384 	mint operator*(mint rhs) { mint res=*this; res*=rhs; return res; }
dd6 	mint operator/(mint rhs) { mint res=*this; res/=rhs; return res; }
7ea 	mint inv() { return this->pow(MOD-2); }
714 	mint pow(int e) {
30b 		mint res(1);
65a 		for(mint p=*this;e>0;e/=2,p*=p) if(e%2)
bbc 			res*=p;
b50 		return res;
f35 	}
b64 };
\end{lstlisting}

\subsection{Polynomial operations}
\begin{lstlisting}
// Multi-point polynomial evaluation: O(n*log^2(n))
// Polynomial interpolation: O(n*log^2(n))
 
// Works with NTT. For FFT, just replace the type.

f66 #define SZ(s) int(s.size())
7e5 using Z = mint<998244353>;
689 typedef vector<Z> poly;

de4 poly add(poly &a, poly &b) {
bea     int n = SZ(a), m = SZ(b);
dba     poly ans(max(n, m));
0cc     for (int i = 0; i < max(n, m); i++) {
d20         if (i < n)
7a8             ans[i] += a[i];
1d6         if (i < m)
229             ans[i] += b[i];
51a     }
277     while (SZ(ans) > 1 && !ans.back().x) ans.pop_back();
ba7     return ans;
3d8 }

dfb poly invert(poly &b, int d) {
5d9     poly c = {b[0].inv ()};
cfd     while (SZ(c) <= d) {
0c0         int j = 2 * SZ(c);
6aa         auto bb = b;
3df         bb.resize(j);
55c         poly cb = NTT::multiply(c, bb);
fbf         for (int i = 0; i < SZ(cb); i++) cb[i] = Z(0) - cb[i];
6d7         cb[0] += 2;
beb         c = NTT::multiply(c, cb);
4cd         c.resize(j);
6bd     }
ea9     c.resize(d + 1);
807     return c;
089 }

b8c pair<poly, poly> divslow(poly &a, poly &b) {
bea     poly q, r = a;
f69     while (SZ(r) >= SZ(b))
f95     {
759         q.pb(r.back() * b.back().inv ());
41f         if (q.back().x)
5c5             for (int i = 0; i < SZ(b); i++)
f95             {
ab0                 r[SZ(r) - i - 1] = r[SZ(r) - i - 1] - q.back() * b[SZ(b) - i - 1];
864             }
515         r.pop_back();
07b     }
bb2     reverse(all(q));
442     return {q, r};
7dc }

958 pair<poly, poly> divide(poly &a, poly &b) { // returns {quotient,remainder}
d01     int m = SZ(a), n = SZ(b), MAGIC = 750;
fbc     if (m < n)
33d         return {{0}, a};
61b     if (min(m - n, n) < MAGIC)
7c0         return divslow(a, b);
64a     poly ap = a; reverse(all(ap));
424     poly bp = b; reverse(all(bp));
160     bp = invert(bp, m - n);
c5b     poly q = NTT::multiply(ap, bp);
63e     q.resize(SZ(q) + m - n - SZ(q) + 1, 0);
bb2     reverse(all(q));
72d     poly bq = NTT::multiply(b, q);
df0     for (int i = 0; i < SZ(bq); i++) bq[i] = Z(0) - bq[i];
b56     poly r = add(a, bq);
442     return {q, r};
224 }

204 vector<poly> tree;

1ee void filltree(vector<Z> &x) {
b3a     int k = SZ(x);
ffb     tree.resize(2 * k);
13c     for (int i = k; i < 2 * k; i++) tree[i] = {Z(0) - x[i - k], 1};
bcd     for (int i = k - 1; i; i--)
9ec         tree[i] = NTT::multiply(tree[2 * i], tree[2 * i + 1]);
6f6 }

591 vector<Z> evaluate(poly &a, vector<Z> &x) {
2b8     filltree(x);
b3a     int k = SZ(x);
a34     vector<poly> ans(2 * k);
8f8     ans[1] = divide(a, tree[1]).second;
7db     for (int i = 2; i < 2 * k; i++) ans[i] = divide(ans[i >> 1], tree[i]).second;
607     vector<Z> r;
c02     for (int i = 0; i < k; i++) r.pb(ans[i + k][0]);
4c1     return r;
f5a }

e05 poly derivate(poly &p) {
c5e     poly ans(SZ(p) - 1);
ff7     for (int i = 1; i < SZ(p); i++) ans[i - 1] = p[i] * i;
ba7     return ans;
c7f }

5a1 poly interpolate(vector<Z> &x, vector<Z> &y) {
2b8     filltree(x);
8c4     poly p = derivate(tree[1]);
3ed     int k = SZ(y);
4fe     vector<Z> d = evaluate(p, x);
888     vector<poly> intree(2 * k);
a0d     for (int i = k; i < 2 * k; i++) intree[i] = {y[i - k] / d[i - k]};
a79     for (int i = k - 1; i; i--) {
a49         poly p1 = NTT::multiply(tree[2 * i], intree[2 * i + 1]);
026         poly p2 = NTT::multiply(tree[2 * i + 1], intree[2 * i]);
452         intree[i] = add(p1, p2);
d48     }
2a3     return intree[1];
ae8 }
\end{lstlisting}



%%%%%%%%%%%%%%%%%%%%
%
% dynamic-programming
%
%%%%%%%%%%%%%%%%%%%%

\section{dynamic-programming}

\subsection{CHT - Dynamic Convex Hull Trick}
\begin{lstlisting}
// Complexity: 
//     add - O(logN)
//     query - O(logN)

// Functions:
//     add(a, b) - add line (a * x + b) to the convex hull.
//     query (x) - return the maximum value of any line on point x.

// Details:
//     If you want to maintain the bottom convex hull, it is
//     easier to just change the sign. Be careful with overflow
//     on query. Can use __int128 to avoid.

72c struct Line {
073     mutable ll a, b, p;
8e3     bool operator<(const Line& o) const { return a < o.a; }
abf     bool operator<(ll x) const { return p < x; }
469 };

326 struct dynamic_hull : multiset<Line, less<>> {
33a     ll div(ll a, ll b) { 
a20         return a / b - ((a ^ b) < 0 and a % b);
a8a     }
        
bbb     void update(iterator x) {
b2a         if (next(x) == end()) x->p = LINF;
772         else if (x->a == next(x)->a) x->p = x->b >= next(x)->b ? LINF : -LINF;
424         else x->p = div(next(x)->b - x->b, x->a - next(x)->a);
0c4     }
    
71c     bool overlap(iterator x) {
f18         update(x);
cfa         if (next(x) == end()) return 0;
a4a         if (x->a == next(x)->a) return x->b >= next(x)->b;
d40         return x->p >= next(x)->p;
901     }
            
176     void add(ll a, ll b) {
1c7         auto x = insert({a, b, 0});
4ab         while (overlap(x)) erase(next(x)), update(x);
dbc         if (x != begin() and !overlap(prev(x))) x = prev(x), update(x);
0fc         while (x != begin() and overlap(prev(x))) 
4d2             x = prev(x), erase(next(x)), update(x);
48f     }
        
4ad     ll query(ll x) {
229         assert(!empty());
7d1         auto l = *lower_bound(x);
d41 #warning cuidado com overflow
aba         return l.a * x + l.b;
3f5     }
8f2 };
\end{lstlisting}



%%%%%%%%%%%%%%%%%%%%
%
% extra
%
%%%%%%%%%%%%%%%%%%%%

\section{extra}

\subsection{Clock}
\begin{lstlisting}
615 double getCurrentTime() {
7dd 	return (double)(clock() - startTime) / CLOCKS_PER_SEC;
0b1 }
\end{lstlisting}

\subsection{hash.sh}
\begin{lstlisting}
d41 # ./hash.sh arquivo.cpp l1 l2
d41 # md5sum do hash.sh: 9cd1295ed4344001c20548b1d6eb55b2
d41 #
d41 # Hash acumulativo, linha por linha:
d41 # for i in $(seq $2 $3); do
d41 #   echo -n "$i "
d41 #   sed -n $2','$i' p' $1 | cpp -dD -P -fpreprocessed | tr -d '[:space:]' | md5sum | cut -c-6
d41 # done
58a sed -n $2','$3' p' $1 | cpp -dD -P -fpreprocessed | tr -d '[:space:]' | md5sum | cut -c-6
\end{lstlisting}

\subsection{Pragmas}
\begin{lstlisting}
827 #pragma GCC target("avx2,bmi,bmi2,lzcnt,popcnt")
\end{lstlisting}

\subsection{Random}
\begin{lstlisting}
836 shuffle(permutation.begin(), permutation.end(), rng);
1ee uniform_int_distribution<int>(a,b)(rng);
\end{lstlisting}

\subsection{Template}
\begin{lstlisting}
ca4 using namespace std;

613 #define all(x) x.begin(), x.end()
5d8 #define int int64_t
efe #define pb push_back

21e void dbg_out() { cerr << endl; }
2dc template <typename H, typename... T>
f62 void dbg_out(H h, T... t) { cerr << ' ' << h; dbg_out(t...); }
89a #define dbg(...) { cerr << #__VA_ARGS__ << ':'; dbg_out(__VA_ARGS__); }

63d void solve() {
fa9 }

0dd signed main(){
9dd 	ios::sync_with_stdio(false); cin.tie(0);
0f9 	solve();
68e }
\end{lstlisting}



%%%%%%%%%%%%%%%%%%%%
%
% graphs
%
%%%%%%%%%%%%%%%%%%%%

\section{graphs}

\subsection{ Euler Walk}
\begin{lstlisting}
//               starting at src. Not necesseraly a cycle. Works for both 
//               directed and undirected. Returns vector 
//               of \{vertex,label of edge to vertex\}.
//               Second element of first pair is always $-1$.
//  Complexity: O(N + M)
//

843 template<bool directed> struct Euler {
a06     using pii = pair<int, int>;
060     int N;
109     vector<vector<pii>> adj; 
1f1     vector<vector<pii>::iterator> its; 
cbd     vector<bool> used;
ee1     Euler (int _N) : N (_N), adj (_N) {}
010     void add_edge(int a, int b) {
e63         int M = used.size (); used.push_back(0); 
215         adj[a].emplace_back(b, M); 
f91         if (!directed) adj[b].emplace_back(a, M);
e01     }
94e     vector<pii> solve(int src = 0) { 
29e         its.resize(N);
3c7         for (int i = 0; i < N; i++) its[i] = begin (adj[i]);
    
805         vector<pii> ans, s{{src,-1}}; // {{vert,prev vert},edge label}
2f5         int lst = -1; // ans generated in reverse order
bdd         while (s.size ()) { 
723             int x = s.back ().first; auto& it=its[x], en=end(adj[x]);
0d5             while (it != en && used[it->second]) ++it;
8af             if (it == en) { // no more edges out of vertex
9c7                 if (lst != -1 && lst != x) return {};
                    // not a path, no tour exists
f10                 ans.push_back(s.back ()); s.pop_back(); 
816                 if (s.size ()) lst=s.back ().first;
38e             } else s.push_back(*it), used[it->second] = 1;
acb         } // must use all edges
0f8         if (ans.size () != used.size () + 1) return {}; 
9ee         reverse(all(ans)); return ans;
340     }
d90 };
\end{lstlisting}

\subsection{ Stable Marriage problem}
\begin{lstlisting}
//  Given n men and n women, where each person has ranked all 
//  members of the opposite sex in order of preference, marry 
//  the men and women together such that there are no two people 
//  of opposite sex who would both rather have each other than 
//  their current partners. When there are no such pairs of 
//  people, the set of marriages is deemed stable.
//
//  If the lists are complete, there is always a solution that
//  can be founc in O(n * m).
//
//  a - Rank list of first group
//  b - Rank list of first group
//  solve () - Gives an stable matching covering the first group.
//             It is necessary that n <= m.
//

7da struct StableMarriage {
14e     int n, m;
fbe     using vvi = vector<vector<int>>;
10e     vvi a, b;
142     StableMarriage (int n, int m, vvi a, vvi b) : n (n), m (m), a (a), b (b) {};
    
201     vector<pair<int, int>> solve () {
5af         assert (n <= m);
d81         vector<int> p (n), mb (m, -1);
8e0         vector rank (m, vector<int> (n));
fd3         for (int i = 0; i < m; i++) for (int j = 0; j < n; j++) rank[i][b[i][j]] = j;
26a         queue<int> q;
    
5af         for (int i = 0; i < n; i++) q.push (i);
402         while (q.size ()) {
be1             int u = q.front (); q.pop ();
    
838             int v = a[u][p[u]++];
af0             if (mb[v] == -1) {
4d0                 mb[v] = u;
36a             }
4e6             else {
b60                 int other_u = mb[v];
a70                 if (rank[v][u] < rank[v][other_u]) {
4d0                     mb[v] = u;
76b                     q.push (other_u);
66b                 }
4e6                 else {
f73                     q.push (u);
366                 }
5ed             }
f7b         }
f77         vector<pair<int, int>> ans;
6e1         for (int i = 0; i < m; i++) if (mb[i] != -1) ans.pb ({mb[i], i});
ba7         return ans;
fe6     }
ab7 };
\end{lstlisting}

\subsection{2-SAT}
\begin{lstlisting}
// Complexity: O(|V| + |E|)
// 
// Functions:
//     either (a, b) - (a | b) is true
//     implies (a, b) - (a -> b) is true
//     must (x) - x is true
//     atMostOne (v) - ensure that at most one of these
//                     variables is true
//     solve () - returns the answer if system is possible.
// 
// Details:
//      Not x is equivalente to ~x on this template.

bf0 struct SCC {
dc0     int N, ti = 0; vector<vector<int>> adj;
70b     vector<int> disc, comp, st, comps;
d5d     void init(int _N) { 
b77         N = _N; 
0f3         adj.resize(N);
9a3         disc.resize(N);
a4e         comp = vector<int>(N,-1); 
d84     }
768     void add_edge(int x, int y) { adj[x].push_back(y); }
e34     int dfs(int x) {
4b4         int low = disc[x] = ++ti; st.push_back(x); // disc[y] != 0 -> in stack
989         for (auto y : adj[x]) if (comp[y] == -1) {
494             auto b = disc[y] ? : dfs(y); auto &a = low;
28a             b < a ? a = b, 1 : 0;
46d         } 
e79         if (low == disc[x]) { // make new SCC, pop off stack until you find x
b3d             comps.push_back(x); for (int y = -1; y != x;) 
e45                 comp[y = st.back()] = x, st.pop_back();
90f         }
b2b         return low;
22e     }
761     void gen() {
50d         for (int i = 0; i < N; i++) if (!disc[i]) dfs(i);
3a5         reverse(all(comps));
592     }
b15 };
 
417 struct TwoSAT {
5ec     int N = 0; vector<pair<int, int>> edges;
8b2     void init (int _N) { N = _N; }
4b3     int addVar () { return N++; }
8c0     void either (int x, int y) { 
8f5         x = max(2 * x, -1 - 2 * x), y = max(2 * y, -1 - 2 * y);
599         edges.push_back ({x, y}); 
c50     }
77e     void implies (int x, int y) { 
7ab         either (~x,y); 
288     }
fa9     void must (int x) { 
f97         either (x,x); 
b95     }
0b6     void atMostOne (const vector<int>& li) {
414         if (li.size () <= 1) return;
da9         int cur = ~li[0];
113         for (int i = 2; i < li.size (); i++) {
b70             int next = addVar();
698             either(cur, ~li[i]); either(cur, next);
0af             either(~li[i], next); cur = ~next;
c0d         }
ed7         either(cur, ~li[1]);
a57     }
28e     vector<bool> solve() {
4ad         SCC S; S.init(2 * N);
d62         for (auto [x, y] : edges) 
7ce             S.add_edge(x ^ 1, y), S.add_edge(y ^ 1, x);
f58         S.gen(); reverse(all(S.comps)); // reverse topo order
76d         for (int i = 0; i < 2 * N; i += 2) 
7bf             if (S.comp[i] == S.comp[i^1]) return {};
586         vector<int> tmp(2 * N); 
6de         for (auto i : S.comps) if (!tmp[i]) 
94d             tmp[i] = 1, tmp[S.comp[i^1]] = -1;
f18         vector<bool> ans(N); 
45f         for (int i = 0; i < N; i++) ans[i] = tmp[S.comp[2*i]] == 1;
ba7         return ans;
b35     }
46a };
\end{lstlisting}

\subsection{Block Cut Tree}
\begin{lstlisting}
// Constructor: SCC(|V|, |E|, [[v, e]; |V|])
// Complexity: O(N+M)

142 struct BlockCutTree {
8d3 	int ncomp; // number of components
f7a 	vector<int> comp; // comp[e]: component of edge e
a1c 	vector<vector<int>> gart; // gart[v]: list of components an articulation point v is adjacent to
    			          // if v is NOT an articulation point, then gart[v] is empty
    				
    	// assumes auto [neighbor_vertex, edge_id] = g[current_vertex][i]
deb 	BlockCutTree(int n, int m, vector<pair<int,int>> g[]): ncomp(0), comp(m), gart(n) {
6bc 		vector<bool> vis(n), vise(m);
594 		vector<int> low(n), prof(n);
46e 		stack<pair<int,int>> st;
    
45f 		function<void(int,bool)> dfs = [&](int v, bool root) {
cca 			vis[v] = 1;
dc9 			int arb = 0; // arborescences
e8a 			for(auto [p, e]: g[v]) if(!vise[e]) {
c8a 				vise[e] = 1;
934 				int in = st.size();
20c 				st.emplace(e, vis[p] ? -1 : p);
137 				if(!vis[p]) {
f07 					arb++;
690 					low[p] = prof[p] = prof[v] + 1;
397 					dfs(p, 0);
de7 					low[v] = min(low[v], low[p]);
23d 				} else low[v] = min(low[v], prof[p]);
c52 				if(low[p] >= prof[v]) {
c80 					gart[v].push_back(ncomp);
080 					while(st.size() > in) {
2b5 						auto [es, ps] = st.top();
8b3 						comp[es] = ncomp;
81d 						if(ps != -1 && !gart[ps].empty())
746 							gart[ps].push_back(ncomp);
25a 						st.pop();
229 					}
a8f 					ncomp++;
f0d 				}
863 			}
7f8 			if(root && arb <= 1) gart[v].clear();
5ee 		};
0f0 		for(int v=0;v<n;v++) if(!vis[v]) dfs(v, 1);
ff8 	}
f70 };
\end{lstlisting}

\subsection{if EDGE is true, the child of the edge represents it}
\begin{lstlisting}
4d4 	vector<int> dad, pos, in, out, h;
	// dad[u]: self-explaining
	// pos[k]: k-th vertex to be visited in dfs order
	// in[u], out[u]: time of visit in dfs of vertex u
	// h[u]: highest ancestor from same hld chain ("head")
	// 	 two vertices u and v are from the same chain iff h[u] == h[v]
2ab 	SEG seg;

585 	HLD(int n, vector<int> g[]): dad(n), pos(n), in(n), out(n), h(n), seg(n) {
1ad 		int t = -1;
214 		function<void(int)> dfs = [&](int u) {
3b3 			pos[ in[u] = ++t ] = u;
c12 			int mx = -1;
e28 			for(int &v: g[u]) if(v != dad[u]) {
2e0 				dad[v] = u;
b3f 				h[v] = g[u][0] == v ? h[u] : v;
6b4 				dfs(v);
ca4 				if(out[v] - in[v] > mx) 
143 					mx = out[v] - in[v], swap(g[u][0], v);
2f4 			}
677 			out[u] = t;
77c 		};
5b8 		dfs(0); t = -1; dfs(0); // yes, twice 
d04 	}

	// if EDGE == true, the child of an edge represents it
67a 	template<typename T>
930 	void updade(int u, T val) {
184 		seg.update(in[u] + EDGE, val);
6be 	}

	// If range update is needed, just replace seg.query with seg.update
658 	template<typename RES> 
5fb 	RES query_path(int u, int v) {
6e5 		RES res = RES();
2bf 		while(h[u] != h[v]) {
dee 			if(in[h[u]] < in[h[v]]) swap(u, v);
6a1 			res = res + seg.query(in[h[u]], in[u]);
d8a 			u = dad[h[u]];
e88 		}
709 		if(in[u] > in[v]) swap(u, v);
    		// u is now the LCA of u and v
912 		if(in[u] + EDGE <= in[v]) 
948 			res = res + seg.query(in[u] + EDGE, in[v]);
b50 		return res;
a37 	}

658 	template<typename RES> 
3ef 	RES query_subtree(int u) {
d83 		if(in[u] + EDGE <= out[u])
fbb 			return seg.query(in[u] + EDGE, out[u]);
4c5 		return RES();
a28 	}
ed7 };
\end{lstlisting}

\subsection{LCA}
\begin{lstlisting}
33e struct LCA {
0ce 	vector<int> pre, dep; // preorder traversal and depth
e16 	RMQ<pair<int,int>> rmq;
    
c67 	LCA() {}
1a3 	LCA(int sz, vector<int> g[], int root): pre(sz), dep(sz) {
837 		vector<pair<int,int>> tour; tour.reserve(2*sz-1);
6be 		auto dfs = [&](int v, int dad, auto& self) -> void {
e17 			pre[v] = tour.size();
95e 			tour.push_back({dep[v],v});
27e 			for(int p: g[v]) if(p != dad) {
5b8 				dep[p] = dep[v]+1;
f5e 				self(p,v,self);
95e 				tour.push_back({dep[v],v});
af6 			}
61f 		};
862 		dfs(root, root, dfs);
b69 		rmq = RMQ<pair<int,int>>(tour);
234 	}
    
4ea 	int query(int a, int b) {
ca7 		if(pre[a] > pre[b]) swap(a,b);
d1b 		return rmq.query(pre[a],pre[b]).second;
f05 	}
    
b5d 	int dist(int a, int b) {
969 		int c = query(a,b);
5a3 		return dep[a] + dep[b] - 2*dep[c];
3de 	}
788 };
\end{lstlisting}

\subsection{Tarjan for undirected graphs}
\begin{lstlisting}
// Constructor: SCC(|V|, |E|, [[v, e]; |V|])
//
// Complexity: O(N+M)

bf0 struct SCC {
27d 	vector<bool> bridge; // bridge[e]: true if edge e is a bridge
f7a 	vector<int> comp; // comp[v]: component of vertex v
    
8d3 	int ncomp; // number of components
1df 	vector<int> sz; // sz[c]: size of component i (number of vertexes)
413 	vector<vector<pair<int, int>>> gc; // gc[i]: list of adjacent components
    				
    	// assumes auto [neighbor_vertex, edge_id] = g[current_vertex][i]
d90 	SCC(int n, int m, vector<pair<int, int>> g[]): bridge(m), comp(n, -1), ncomp(0) {
5c8 		vector<bool> vis(n);
594 		vector<int> low(n), prof(n);
    
208 		function<void(int,int)> dfs = [&](int v, int dad) {
cca 			vis[v] = 1;
290 			for(auto [p, e]: g[v]) if(p != dad) {
137 				if(!vis[p]) {
690 					low[p] = prof[p] = prof[v] + 1;
345 					dfs(p, v);
de7 					low[v] = min(low[v], low[p]);
c9b 				} else low[v] = min(low[v], prof[p]);
edd 			}
3f2 			if(low[v] == prof[v]) ncomp++;
729 		};
548 		for(int i=0;i<n;i++) if(!vis[i]) dfs(i, -1);
    
7cc 		sz.resize(ncomp); gc.resize(ncomp);
    
ac9 		int cnt = 0;
c64 		function<void(int,int)> build = [&](int v, int c) {
440 			if(low[v] == prof[v]) c = cnt++;
d5f 			comp[v] = c;
24a 			sz[c]++;
936 			for(auto [p, e]: g[v]) if(comp[p] == -1) {
5e7 				build(p, c);
a54 				int pc = comp[p];
d59 				if(c != pc) {
442 					bridge[e] = true;
718 					gc[c].emplace_back(pc, e);
2a3 					gc[pc].emplace_back(c, e);
b6e 				}
cf9 			}
731 		};
c7d 		for(int i=0;i<n;i++) if(comp[i] == -1) build(i, -1);
561 	}
a1e };
\end{lstlisting}

\subsection{Virtual Tree}
\begin{lstlisting}
f03 namespace vtree {
dbb 	vector<int> vg[MAX];
    
    	// receives list of vertexes and returns root of virtual tree
    	// v must NOT be empty
cf3 	int build(vector<int> vs, LCA& lca) {
aa3 		auto cmp = [&](int i, int j) {
d31 			return lca.pre[i] < lca.pre[j];
645 		};
de1 		sort(all(vs), cmp);
7b1 		for(int i=vs.size()-1; i>0; i--) vs.push_back(lca.query(vs[i-1], vs[i]));
47a 		sort(all(vs));
f7c 		vs.resize(unique(all(vs))-vs.begin());
de1 		sort(all(vs), cmp);
a9f 		for(auto v: vs) vg[v].clear();
ab1 		for(int i=1;i<vs.size();i++) {
258 			int dad = lca.query(vs[i-1], vs[i]);
993 			vg[dad].push_back(vs[i]);
d85 			vg[vs[i]].push_back(dad);
d34 		}
367 		return vs[0];
373 	}
ea9 }
\end{lstlisting}



%%%%%%%%%%%%%%%%%%%%
%
% geometry
%
%%%%%%%%%%%%%%%%%%%%

\section{geometry}

\subsection{Convex Hull}
\begin{lstlisting}
// Returns in CCW order (reversed in x in UPPER)
// Complexity: O(NlogN)

9c0 template <bool UPPER>
6d8 vector<point> hull(vector<point> v) {
805 	vector<point> res;
6cd 	if(UPPER) for(auto& p: v) p.x = -p.x, p.y = -p.y;
304 	sort(all(v));
3f5 	for(auto& p: v) {
1e7 		if(res.empty()) { res.push_back(p); continue; }
89e 		if(res.back().x == p.x) continue;
ca3 		while(res.size() >= 2) {
dd1 			point a = res[res.size()-2], b = res.back();
039 			if(!left(a, b, p)) res.pop_back();
    			//to include collinear points
    			//if(right(a, b, p)) res.pop_back();
f97 			else break;
d33 		}
6f7 		res.push_back(p);
806 	}
96b 	if(UPPER) for(auto& p: res) p.x = -p.x, p.y = -p.y;
b50 	return res;
72a }
\end{lstlisting}

\subsection{Double geometry}
\begin{lstlisting}

ad8 constexpr double EPS = 1e-10;

664 bool zero(double x) {
efc 	return abs(x) <= EPS;
e8f }

// CORNER: point = (0, 0)
be5 struct point {
662 	double x, y;
    	
5cb 	point(): x(), y() {}
581 	point(double _x, double _y): x(_x), y(_y) {}
    	
587 	point operator+(point rhs) { return point(x+rhs.x, y+rhs.y); }
2f1 	point operator-(point rhs) { return point(x-rhs.x, y-rhs.y); }
df3 	point operator*(double k) { return point(x*k, y*k); }
d22 	point operator/(double k) { return point(x/k, y/k); }
027 	double operator*(point rhs) { return x*rhs.x + y*rhs.y; }
c47 	double operator^(point rhs) { return x*rhs.y - y*rhs.x; }
    
aa4 	point rotated(point polar) { return point(*this^polar,*this*polar); }
b9a 	point rotated(double ang) { return (*this).rotated(point(sin(ang),cos(ang))); }
b7c 	double norm2() { return *this * *this; }
b3a 	double norm() { return sqrt(norm2()); }
    
5fa 	bool operator<(const point& rhs) const {
70b 		return x < rhs.x - EPS || (zero(x-rhs.x) && y < rhs.y - EPS);
f87 	}
    
bfa 	bool operator==(const point& rhs) const {
d38 		return zero(x-rhs.x) && zero(y-rhs.y);
4f7 	}
71f };

e17 const point ccw90(1, 0), cw90(-1, 0);

// angular comparison in [0, 2pi)
// smallest is (1, 0)
// CORNER: a || b == (0, 0)
a43 bool ang_cmp(point a, point b) {
b41 	auto quad = [](point p) -> bool {
    		// 0 if ang in [0, pi), 1 if in [pi, 2pi)
cfb 		return p.y < 0 || (p.y == 0 && p.x < 0);
428 	};
028 	using tup = tuple<bool, double>;
dab 	return tup{quad(a), 0} < tup{quad(b), a^b};
7d8 }

b5e double dist2(point p, point q) { // squared distance
f70     return (p - q)*(p - q);
60f }

cf4 double dist(point p, point q) {
d92     return sqrt(dist2(p, q));
a75 }

70f double area2(point a, point b, point c) { // two times signed area of triangle abc
b44 	return (b - a) ^ (c - a);
556 }

97b bool left(point a, point b, point c) {
f3e 	return area2(a, b, c) > EPS; // counterclockwise
483 }

18a bool right(point a, point b, point c) {
682 	return area2(a, b, c) < -EPS; // clockwise
cc2 }

62c bool collinear(point a, point b, point c) {
56f 	return zero(area2(a,b,c));
16b }

// CORNER: a || b == (0, 0)
e00 int parallel(point a, point b) {
046 	if(!zero(a ^ b)) return 0;
8bb 	return (a.x>0) == (b.x>0) && (a.y > 0) == (b.y > 0) ? 1 : -1;
e6c }

// CORNER: a == b
565 struct segment {
393 	point a, b;
    
889 	segment() {}
e93 	segment(point _a, point _b): a(_a), b(_b) {}
    
3a8 	point vec() { return b - a; }
    
641 	bool contains(point p) {
969 		return a == p || b == p || parallel(a-p, b-p) == -1;
e3e 	}
    
e82 	point proj(point p) { // projection of p onto segment
983 		p = p - a;
382 		point v = vec();
cdd 		return a + v*((p*v)/(v*v));
bd9 	}
    
c2b };

e58 bool intersects(segment r, segment s) {
19c 	if(r.contains(s.a) || r.contains(s.b) || s.contains(r.a) || s.contains(r.b)) return 1;
9ff 	return left(r.a, r.b, s.a) != left(r.a, r.b, s.b) && 
0a2 		left(s.a, s.b, r.a) != left(s.a, s.b, r.b);
99a }

6cc bool parallel(segment r, segment s) {
3c2 	return parallel(r.vec(), s.vec());
407 }

737 point line_intersection(segment r, segment s) {
2de 	if(parallel(r, s)) return point(HUGE_VAL, HUGE_VAL);
3b1 	point vr = r.vec(), vs = s.vec();
68c 	double cr = vr ^ r.a, cs = vs ^ s.a;
47e 	return (vs*cr - vr*cs) / (vr ^ vs);
2d1 }

d2f struct polygon {
768 	vector<point> vp;
1a8 	int n;
    
0a4 	polygon(vector<point>& _vp): vp(_vp), n(vp.size()) {
732 		if(area2() < -EPS) reverse(all(_vp));
209 	}
    
a2f 	int nxt(int i) { return i+1<n ? i+1 : 0; }
6af 	int prv(int i) { return i ? i-1 : 0; }
    
    	// If positive, the polygon is in ccw order. It is in cw order otherwise.
0d8 	double area2() { // O(n
a2e 		double acum = 0;
830 		for(int i = 0; i < n; i++)
159 			acum += vp[i] ^ vp[nxt(i)];
a13 		return acum;
704 	}
    
223 	bool has(point p) { // O(log n). The polygon must be convex and in ccw order
947 		if(right(vp[0], vp[1], p) || left(vp[0], vp[n-1], p)) return 0;
9da 		int lo = 1, hi = n;
3d1 		while(lo + 1 < hi) {
c86 			int mid = (lo + hi) / 2;
395 			if(!right(vp[0], vp[mid], p)) lo = mid;
8c0 			else hi = mid;
a27 		}
b27 		return hi != n ? !right(vp[lo], vp[hi], p) : dist2(vp[0], p) < dist2(vp[0], vp[n-1]) + EPS;
8fe 	}
    
8d5 	double calipers() { // O(n). The polygon must be convex and in ccw order.
e9c 		double ans = 0;
1ed 		for(int i = 0, j = 1; i < n; i++) {
d97 			point v = vp[nxt(i)] - vp[i];
d5f 			while((v ^ (vp[nxt(j)] - vp[j])) > EPS) j = nxt(j);
    			// do something with vp[i] and vp[j]
e88 			ans = max(ans, dist2(vp[i], vp[j])); // Example with polygon diameter squared
121 		}
ba7 		return ans;
63b 	}
    
    	// returns the maximal point using comparator cmp
    	// example: 
    	// 	extreme([&](point p, point q) {return p * v > q * v;});
    	// 	returns point with maximal dot product with v
8ff 	int extreme(const function<bool(point, point)> &cmp) {
121 		auto is_extreme = [&](int i, bool& cur_dir) -> bool {
ec0 			cur_dir = cmp(vp[nxt(i)], vp[i]);
e14 			return !cmp(vp[prv(i)], vp[i]) && !cur_dir;
8fc 		};
63d 		bool last_dir, cur_dir;
b89 		if(is_extreme(0, last_dir)) return 0;
a04 		int lo = 0, hi = n; 
3d1 		while(lo + 1 < hi) {
591 			int m = (lo + hi) / 2;
dcc 			if(is_extreme(m, cur_dir)) return m;
983 			bool rel_dir = cmp(vp[m], vp[lo]);
8ed 			if((!last_dir && cur_dir) || (last_dir == cur_dir && rel_dir == cur_dir)) {
04a 				lo = m;
1f1 				last_dir = cur_dir;
8d6 			} else hi = m;
c26 		}
253 		return lo;
1a5 	}
    
6fb 	pair<int, int> tangent(point p) { // O(log n) for convex polygon in ccw orientation
    		// Finds the indices of the two tangents to an external point q
c71 		auto left_tangent = [&](point r, point s) -> bool {
f70 			return right(p, r, s);
854 		};
b40 		auto right_tangent = [&](point r, point s) -> bool {
f88 			return left(p, r, s);
06e 		};
6b0 		return {extreme(left_tangent), extreme(right_tangent)};
7b8 	}
    
df5 	void normalize() { // p[0] becomes the lowest leftmost point 
b2f 		rotate(vp.begin(), min_element(all(vp)), vp.end());
7e8 	}
    
0da 	polygon operator+(polygon& rhs) { // Minkowsky sum
335 		normalize();
61f 		rhs.normalize();
244 		vector<point> sum;
ccc 		double dir;
337 		for(int i = 0, j = 0; i < n || j < rhs.n; i += dir > -EPS, j += dir < EPS) {
c6f 			sum.push_back(vp[i % n] + rhs.vp[j % rhs.n]);
727 			dir = (vp[(i + 1) % n] - vp[i % n]) 
59c 				^ (rhs.vp[(j + 1) % rhs.n] - rhs.vp[j % rhs.n]);
d98 		}
6b4 		return polygon(sum);
319 	}
c25 };

// Circle
//  Basic structure of circle and operations related with it.
// 
// All operations' time complexity are O(1)

1d5 const double PI = acos(-1);

aa8 struct circle {
664 	point o; double r;
    
d0b 	circle() {}
187 	circle(point _o, double _r) : o(_o), r(_r) {}
    	// CORNER CASE: a, b and c must NOT be collinear
26a 	circle(point a, point b, point c) {
f07 		b = b - a;
0e8 		c = c - a;
a23 		double B = b.norm2();
d09 		double C = c.norm2();
158 		double D = b ^ c;
d58 		o = a + point( (c.y*B - b.y*C) / (2*D), (b.x * C - c.x * B) / (2*D) );
5ed 		r = (o-a).norm();
9ae 	}
    
223 	bool has(point p) { 
804 		return (o - p).norm2() < r*r + EPS;
003 	}
8b0 	vector<point> operator/(circle c) { // Intersection of circles.
4b4 		vector<point> inter;                   // The points in the output are in ccw order.
6ac 		double d = (o - c.o).norm();
376 		if(r + c.r < d - EPS || d + min(r, c.r) < max(r, c.r) - EPS)
21d 			return {};
ea5 		double x = (r*r - c.r*c.r + d*d) / (2*d);
260 		double y = sqrt(r*r - x*x);
5e0 		point v = (c.o - o) / d;
4be 		inter.pb(o + v*x + v.rotated(cw90)*y);
428 		if(y > EPS) inter.pb(o + v*x + v.rotated(ccw90)*y);
c17 		return inter;
ad5 	}
196 	vector<point> tang(point p){
593 		double d = sqrt(dist2(p, o) - r*r);
164 		return *this / circle(p, d);
7f7 	}
641 	bool contains(point p){ // non strictly inside
fe5 		double d = dist2(o, p);
765 		return d < r * r + EPS;
6be 	}
d11 };
\end{lstlisting}

\subsection{Half-plane intersection}
\begin{lstlisting}
// empty or a convex polygon (maybe degenerated). This template depends on double.cpp
//
// h - (input) set of half-planes to be intersected. Each half-plane is described as a pair
// of points such that the half-plane is at the left of them.
// pol - the intersection of the half-planes as a vector of points. If not empty, these
// points describe the vertices of the resulting polygon in clock-wise order.
// WARNING: Some points of the polygon might be repeated. This may be undesirable in some
// cases but it's useful to distinguish between empty intersections and degenerated
// polygons (such as a point, line, segment or half-line).
//
// Time complexity: O(n logn)

7a9 struct halfplane: public segment {
fe9 	double ang;
077 	halfplane() {}
7c9 	halfplane(point _a, point _b) {
cab 		a = _a; b = _b;
a36 		ang = atan2(v().y, v().x);
461 	}
535 	bool operator <(const halfplane& rhs) const {
287 		if (fabsl(ang - rhs.ang) < EPS) return right(a, b, rhs.a);
004 	        return ang < rhs.ang;
576 	}
3b2 	bool operator ==(const halfplane& rhs) const {
a0f 		return fabs(ang - rhs.ang) < EPS; 
745 	}
83c 	bool out(point r) {
ad7 		return right(a, b, r);
6ae 	}
485 };

7d1 constexpr double INF = 1e19;
0cd vector<point> hp_intersect(vector<halfplane> h) {
a85 	array<point, 4> box = {
765 		point(-INF, -INF),
822 		point(INF, -INF),
ac0 		point(INF, INF),
006 		point(-INF, INF),
9bb 	};
c63 	for(int i = 0; i < 4; i++)
e4b 		h.emplace_back(box[i], box[(i+1) % 4]);
d77 	sort(all(h));
b1b 	h.resize(unique(all(h)) - h.begin());
ff6 	deque<halfplane> dq;
    
c76 	auto sz = [&]() -> int { return dq.size(); };
    
6e3 	for(auto hp: h) {
673 		while(sz() > 1 && hp.out(line_intersection(dq.back(), dq[sz() - 2])))
c70 			dq.pop_back();
70c 		while(sz() > 1 && hp.out(line_intersection(dq[0], dq[1])))
c68 			dq.pop_front();
1d5 		dq.push_back(hp);
34d 	}
a26 	while(sz() > 2 && dq[0].out(line_intersection(dq.back(), dq[sz() - 2])))
c70 		dq.pop_back();
430 	while(sz() > 2 && dq.back().out(line_intersection(dq[0], dq[1])))
c68 		dq.pop_front();
040 	if(sz() < 3) return {};
e5f 	vector<point> pol(sz());
21d 	for(int i = 0; i < sz(); i++) {
3bb 		pol[i] = line_intersection(dq[i], dq[(i+1) % sz()]);
39e 	}
b22 	return pol;
7c5 }
\end{lstlisting}

\subsection{Integer Geometry}
\begin{lstlisting}

8d0 bool zero(int x) {
5db 	return x == 0;
9b6 }

// CORNER: point = (0, 0)
be5 struct point {
e91 	int x, y;
    	
5cb 	point(): x(), y() {}
4b6 	point(int _x, int _y): x(_x), y(_y) {}
    	
587 	point operator+(point rhs) { return point(x+rhs.x, y+rhs.y); }
2f1 	point operator-(point rhs) { return point(x-rhs.x, y-rhs.y); }
f24 	int operator*(point rhs) { return x*rhs.x + y*rhs.y; }
55a 	int operator^(point rhs) { return x*rhs.y - y*rhs.x; }
    
950 	int norm2() { return *this * *this; }
    
e1c 	using tup = tuple<int, int>;
    
5fa 	bool operator<(const point& rhs) const {
046 		return tup{x, y} < tup{rhs.x, rhs.y};
4a4 	}
    	
bfa 	bool operator==(const point& rhs) const {
024 		return tup{x, y} == tup{rhs.x, rhs.y};
77f 	}
5ad };

// angular comparison in [0, 2pi)
// smallest is (1, 0)
// CORNER: a || b == (0, 0)
a43 bool ang_cmp(point a, point b) {
b41 	auto quad = [](point p) -> bool {
    		// 0 if ang in [0, pi), 1 if in [pi, 2pi)
cfb 		return p.y < 0 || (p.y == 0 && p.x < 0);
428 	};
c41 	using tup = tuple<bool, int>;
dab 	return tup{quad(a), 0} < tup{quad(b), a^b};
401 }

4c6 int dist2(point p, point q) { // squared distance
f70     return (p - q)*(p - q);
288 }

5bf int area2(point a, point b, point c) { // two times signed area of triangle abc
b44 	return (b - a) ^ (c - a);
214 }

97b bool left(point a, point b, point c) {
8a5 	return area2(a, b, c) > 0; // counterclockwise
8fd }

18a bool right(point a, point b, point c) {
c85 	return area2(a, b, c) < 0; // clockwise
ece }

62c bool collinear(point a, point b, point c) {
56f 	return zero(area2(a,b,c));
16b }

// CORNER: a || b == (0, 0)
e00 int parallel(point a, point b) {
046 	if(!zero(a ^ b)) return 0;
8bb 	return (a.x>0) == (b.x>0) && (a.y > 0) == (b.y > 0) ? 1 : -1;
e6c }

// CORNER: a == b
565 struct segment {
393 	point a, b;
    	
877 	segment(): a(), b() {}
e93 	segment(point _a, point _b): a(_a), b(_b) {}
    
3a8 	point vec() { return b - a; }
    
641 	bool contains(point p) {
969 		return a == p || b == p || parallel(a-p, b-p) == -1;
e3e 	}
633 };

e58 bool intersects(segment r, segment s) {
2fb 	if(contains(r, s.a) || contains(r, s.b) || contains(s, r.a) || contains(s, r.b)) return 1;
9ff 	return left(r.a,r.b,s.a) != left(r.a,r.b,s.b) && 
0a2 		left(s.a, s.b, r.a) != left(s.a, s.b, r.b);
3dc }

6cc bool parallel(segment r, segment s) {
3c2 	return parallel(r.vec(), s.vec());
407 }

d2f struct polygon {
768 	vector<point> vp;
1a8 	int n;
    
0a4 	polygon(vector<point>& _vp): vp(_vp), n(vp.size()) {
3f5 		if(area2() < 0) reverse(all(_vp));
256 	}
    
a2f 	int nxt(int i) { return i+1<n ? i+1 : 0; }
6af 	int prv(int i) { return i ? i-1 : 0; }
    
    	// If positive, the polygon is in ccw order. It is in cw order otherwise.
82b 	int area2() { // O(n
745 		int acum = 0;
830 		for(int i = 0; i < n; i++)
159 			acum += vp[i] ^ vp[nxt(i)];
a13 		return acum;
82c 	}
    
223 	bool has(point p) { // O(log n). The polygon must be convex and in ccw order
947 		if(right(vp[0], vp[1], p) || left(vp[0], vp[n-1], p)) return 0;
9da 		int lo = 1, hi = n;
3d1 		while(lo + 1 < hi) {
c86 			int mid = (lo + hi) / 2;
395 			if(!right(vp[0], vp[mid], p)) lo = mid;
8c0 			else hi = mid;
a27 		}
78d 		return hi != n ? !right(vp[lo], vp[hi], p) : dist2(vp[0], p) <= dist2(vp[0], vp[n-1]);
aa8 	}
    
be9 	int calipers() { // O(n). The polygon must be convex and in ccw order.
1a4 		int ans = 0;
1ed 		for(int i = 0, j = 1; i < n; i++) {
d97 			point v = vp[nxt(i)] - vp[i];
775 			while((v ^ (vp[nxt(j)] - vp[j])) > 0) j = nxt(j);
    			// do something with vp[i] and vp[j]
e88 			ans = max(ans, dist2(vp[i], vp[j])); // Example with polygon diameter squared
c95 		}
ba7 		return ans;
e14 	}
    
    	// returns the maximal point using comparator cmp
    	// example: 
    	// 	extreme([&](point p, point q) {return p * v > q * v;});
    	// 	returns point with maximal dot product with v
8ff 	int extreme(const function<bool(point, point)> &cmp) {
121 		auto is_extreme = [&](int i, bool& cur_dir) -> bool {
ec0 			cur_dir = cmp(vp[nxt(i)], vp[i]);
e14 			return !cmp(vp[prv(i)], vp[i]) && !cur_dir;
8fc 		};
63d 		bool last_dir, cur_dir;
b89 		if(is_extreme(0, last_dir)) return 0;
a04 		int lo = 0, hi = n; 
3d1 		while(lo + 1 < hi) {
591 			int m = (lo + hi) / 2;
dcc 			if(is_extreme(m, cur_dir)) return m;
983 			bool rel_dir = cmp(vp[m], vp[lo]);
8ed 			if((!last_dir && cur_dir) || (last_dir == cur_dir && rel_dir == cur_dir)) {
04a 				lo = m;
1f1 				last_dir = cur_dir;
8d6 			} else hi = m;
c26 		}
253 		return lo;
1a5 	}
    
6fb 	pair<int, int> tangent(point p) { // O(log n) for convex polygon in ccw orientation
    		// Finds the indices of the two tangents to an external point q
c71 		auto left_tangent = [&](point r, point s) -> bool {
f70 			return right(p, r, s);
854 		};
b40 		auto right_tangent = [&](point r, point s) -> bool {
f88 			return left(p, r, s);
06e 		};
6b0 		return {extreme(left_tangent), extreme(right_tangent)};
7b8 	}
    
df5 	void normalize() { // p[0] becomes the lowest leftmost point 
b2f 		rotate(vp.begin(), min_element(all(vp)), vp.end());
7e8 	}
    
0da 	polygon operator+(polygon& rhs) { // Minkowsky sum
335 		normalize();
61f 		rhs.normalize();
244 		vector<point> sum;
755 		for(int i = 0, j = 0, dir; i < n || j < rhs.n; i += dir >= 0, j += dir <= 0) {
c6f 			sum.push_back(vp[i % n] + rhs.vp[j % rhs.n]);
727 			dir = (vp[(i + 1) % n] - vp[i % n]) 
59c 				^ (rhs.vp[(j + 1) % rhs.n] - rhs.vp[j % rhs.n]);
520 		}
6b4 		return polygon(sum);
5df 	}
223 };
\end{lstlisting}

\subsection{Minimum Enclosing Circle}
\begin{lstlisting}
// Complexity: Randomized O(N), with constant factor of 10
01f circle mec(vector<point> vp) {
    	//shuffle(all(vp), rng); NECESSARY IF NOT ALREADY SHUFFLED
ad9 	circle c(vp[0], 0);
be9 	auto &[o, r] = c;
694 	for(int i=0;i<int(vp.size());i++) if (!c.contains(vp[i])) {
a58 		o = vp[i], r = 0;
c7c 		for(int j=0;j<i;j++) if (!c.contains(vp[j])) {
425 			o = (vp[i] + vp[j]) / 2;
480 			r = dist(o, vp[i]);
eaf 			for(int k=0;k<j;k++) if(!c.contains(vp[k]))
11e 				c = circle(vp[i], vp[j], vp[k]);
fb6 		}
e19 	}
807 	return c;
e95 }
\end{lstlisting}

\subsection{Nearest Points}
\begin{lstlisting}
// Complexity: O(NlogN)

505 template <typename C_T>
e26 C_T nearest_points(vector<point> v) {
695 	using lim = numeric_limits<C_T>;
50a 	C_T res = lim::max(), sq = sqrt((double)res);
304 	sort(all(v));
6e3 	for(int i=1;i<v.size();i++) if(v[i] == v[i-1]) return 0;
e54 	auto by_y = [](const point& a, const point& b) {
c0c 		using tup = tuple<C_T, C_T>;
1b4 		return tup{a.y, a.x} < tup{b.y, b.x};
58e 	};
aa9 	queue<point> active;
252 	set<point, decltype(by_y)> pts(by_y);
3f5 	for(auto& p: v) {
c24 		while(!active.empty() && p.x-active.front().x > sq) {
56c 			pts.erase(active.front());
1a0 			active.pop();
ab0 		}
abd 		auto it = pts.lower_bound({lim::min(), p.y-sq});
97f 		while(it != pts.end() && it->y <= p.y + sq) {
6fc 			C_T d = dist2(p, *it);
424 			if(d < res) {
b9f 				res = d;
a2c 				sq = sqrt((double)res);
bc7 			}
40d 			it++;
16e 		}
381 		active.push(p);
aa4 		pts.insert(p);
367 	}
b50 	return res;
558 }
\end{lstlisting}

\subsection{Shamos Hoey}
\begin{lstlisting}
// SEGMENTOS NÃO DEVEM SER DEGENERADOS
//
// Checa se existem segmentos que se intersectam
// Complexidade: O(N logN)

4d0 bool shamos_hoey(vector<segment> seg) {
    	// create sweep segment events {x, type, seg_id}
900 	vector<tuple<point, bool, int>> ev;
071 	for(int i=0; i<seg.size(); i++) {
035 		if(seg[i].b < seg[i].a) swap(seg[i].a, seg[i].b);
4ed 		ev.emplace_back(seg[i].a, 0, i);
d2a 		ev.emplace_back(seg[i].b, 1, i);
3d7 	}
075 	sort(all(ev));
    	// CORNER CASE: r.a == s.a && collinear(r.a, r,b, s.b) 
    	// cmp(r, s) == cmp(s, r) => r == s !!!
2e7 	auto cmp = [](segment r, segment s) -> bool {
6c3 		if(r.a == s.a) return left(r.a, r.b, s.b);
4c1 		else if(r.a < s.a) return left(r.a, r.b, s.a);
8ec 		else return !left(s.a, s.b, r.a);
6ab 	};
91a 	set<segment, decltype(cmp)> s(cmp);
2af 	for(auto [_, b, id]: ev) {
4ea 		segment at = seg[id];
22d 		if(!b) {
8c2 			auto nxt = s.lower_bound(at);
556 			if((nxt != s.end() && intersects(*nxt, at))
0b1 				|| (nxt != s.begin() && intersects(*prev(nxt), at)))
6a5 					return 1;
9be 			s.insert(at);
9d9 		} else {
381 			auto cur = s.find(at);
f0e 			if(cur != s.begin() && next(cur) != s.end() && 
38b 					intersects(*prev(cur), *next(cur))) 
6a5 				return 1;
50d 			s.erase(at);
481 		}
94d 	}
bb3 	return 0;
7d0 }
\end{lstlisting}

\pagebreak


%%%%%%%%%%%%%%%%%%%%
%
% Extra
%
%%%%%%%%%%%%%%%%%%%%

\section{Extra}

\subsection{template.cpp}
\begin{lstlisting}
// Template
#include <bits/stdc++.h>
using namespace std;

#define all(x) x.begin(), x.end()
#define int int64_t
#define pb push_back

void dbg_out() { cerr << endl; }
template <typename H, typename... T>
void dbg_out(H h, T... t) { cerr << ' ' << h; dbg_out(t...); }
#define dbg(...) { cerr << #__VA_ARGS__ << ':'; dbg_out(__VA_ARGS__); }

void solve() {
}

signed main(){
	ios::sync_with_stdio(false); cin.tie(0);
	solve();
}
\end{lstlisting}

\subsection{hash.sh}
\begin{lstlisting}
# hash.sh
# Para usar (hash das linhas [l1, l2]):
# ./hash.sh arquivo.cpp l1 l2
# md5sum do hash.sh: 9cd1295ed4344001c20548b1d6eb55b2
#
# Hash acumulativo, linha por linha:
# for i in $(seq $2 $3); do
#   echo -n "$i "
#   sed -n $2','$i' p' $1 | cpp -dD -P -fpreprocessed | tr -d '[:space:]' | md5sum | cut -c-6
# done
sed -n $2','$3' p' $1 | cpp -dD -P -fpreprocessed | tr -d '[:space:]' | md5sum | cut -c-6
\end{lstlisting}

\subsection{random.cpp}
\begin{lstlisting}
// Random
mt19937_64 rng(chrono::steady_clock::now().time_since_epoch().count());
shuffle(permutation.begin(), permutation.end(), rng);
uniform_int_distribution<int>(a,b)(rng);
\end{lstlisting}

\subsection{clock.cpp}
\begin{lstlisting}
// Clock
clock_t startTime = clock();
double getCurrentTime() {
	return (double)(clock() - startTime) / CLOCKS_PER_SEC;
}
\end{lstlisting}

\subsection{pragma.cpp}
\begin{lstlisting}
// Pragmas
#pragma GCC optimize("O3,unroll-loops")
#pragma GCC target("avx2,bmi,bmi2,lzcnt,popcnt")
\end{lstlisting}

\end{document}
